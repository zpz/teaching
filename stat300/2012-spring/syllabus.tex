\documentclass{article}
\usepackage{techart}
\parindent=0pt
\parskip=4pt
\usepackage{ulem}

\begin{document}
\textbf{\large STAT F300 \hfill Spring 2012\\
Statistics}

MWF 9:15--10:15 Chapman 104

\bigskip
\textbf{\large Instructor}: Zepu Zhang

Contact: Chapman 306A; zzhang6@alaska.edu;
    474-7605; mail box in Chapman 101.

Office Hours: 3:00--5:00 PM Mondays and Thursdays.

Also, USE the Math Lab located in Chapman 305.
(You may need to make several visits until you identify
a particular student tutor/worker whose style and expertise match your
need the best; then, find out this tutor's work schedule in the Math Lab
and keep them busy at work!)


\bigskip
\textbf{\large Required Textbook}:
\textit{Probability and Statistics for Engineering and the Sciences},
by Jay L.\@ Devore, 8th ed. Published by Brooks/Cole Cengage Learning, 2009.

The 7th ed (2004) is an acceptable substitute.
Differences between the two editions are minor.

The first two chapters of the book
are available at the course Blackboard website
for your convenience in the first lectures.
Do not defer the decision to buy the book because of this!
You need to have the book throughout the course.

We plan to cover chapters 1--8 and 12, except

{\obeylines
1.2: `stem-and-leaf displays'
1.3: `categorical data and sample proportions'
2.5: `Independence of more than two events'
3.4: `Using Binomial tables' (examples in this section are required; it's only `binomial table look-up' that's not required)
3.5: entire section
3.6: entire section
4.4: `Gamma function', `Gamma distribution', `Chi-squared distribution'
4.5: entire section
4.6: `Beyond normality'
5.4: `Other applications of the CLT'
6.1: after Example 6.10
6.2: `The Method of moments'
7.1: `Bootstrap confidence intervals'
7.2: `A general large-sample confidence interval' and after
7.3: `A prediction interval for a single future value' and after
8.3: entire section
8.5: entire section
12.4: entire section
12.5: `Inferences about the population correlation coefficient' and after
}


\bigskip
\textbf{\large Goals and expected learning outcomes:}
This is a calculus-based introductory statistical course emphasizing
applications. Topics include probability, joint and conditional
probability, expectation and variance, maximum likelihood,
hypothesis tests, and simple linear regression.
Students will understand these topics and learn their applications with
the help of computer software.

\bigskip
\textbf{\large Prerequisites:}
MATH 200, 262, 272, or an equivalent course (i.e.\@ calculus I).

\emph{Note}: a student may not use STAT~200 and STAT~300 to meet the
requirement of a year's sequence course in statistics.

\emph{Stat 300 vs 200}: these two courses cover similar topics; 300 requires
calculus whereas 200 does not; 300 is mathematically more `mature'. If
you have taken 200, usually
you do not need to take 300---the jump is not large enough.


\bigskip
\textbf{\large Blackboard}:
A Blackboard course site has been set up. You will use it to access
announcements,
homework assignments and solutions,
lecture notes,
grades, and
other materials.
Be sure to check out the Blackboard announcements periodically,
say at least once a week.

\bigskip
\textbf{\large Computing:}
Most computational tasks in this course involve elementary arithmetics,
which you can perform by hand with the help of a calculator or computer.

Some tasks involve differentiation and integration.
\emph{Derivatives and integrals must be derived by hand.}
You are not allowed to use computer software, e.g.\@ Mathematica,
to do `symbolic math', e.g.\@ integration.

We will use the free software \texttt{R} for demonstrations and
computations, including graphics and some standard statistical
functions.
The software can be downloaded at \url{http:///www.r-project.org}.
However, the web interface \url{http://cloudst.at} (CloudStat)
seems to suffice for our purposes;
in that case, you do not need to install \texttt{R} on your computer.

Some materials are posted on Blackboard to help you get started with
\texttt{R}. Some lecture time will be used to introduce and help with
\texttt{R}.

In your homework you will not need to show computer code.

\bigskip
\textbf{\large Homework, quizzes, and exams}

There will be 6--7 homework assignments due on designated days
\emph{in class or by 4pm in the instructor's mailbox in Chapman 101}.
Turn in hard copies only.
Remember to number the pages and use a stapler.

\emph{Homework corrections}
After you get your graded homework back, you are encouraged to
turn in detailed corrections within a week.
The corrections should be done separately from your original work
(i.e.\@ not ``in-place'') and should be a clean and complete
re-do of the involved steps, sub-problems, or entire problems.
Include an explanation of how you got wrong, e.g.\@ computational
error, conceptual error, formula copied wrong, etc.
Staple the corrections to the front of your original submission,
and mark ``Corrections''.
\emph{Well-done corrections will get you up to half of the lost
points.}

There will be short in-class quizzes on most Mondays, taking up to 20
minutes. Your answers may be collected, but will not be graded.

There will be 1 in-class hour-long midterm exam, planned right before
the spring break.

The final exam is 2-hours long.
The final exam could be comprehensive.

The midterm and final exams will be \emph{closed-book}.
You may bring \emph{notes on 4 sheets} of double-sided
$8 \frac{1}{2} \times 11$ inch paper prepared by yourself
(as opposed to some cheat sheet found online).
You may not use a computer.
You will need a calculator.

See the ``Tentative Schedule'' for details.

\bigskip
\textbf{\large Grading}

\emph{Late homework will not be accepted in general.}
Exceptions are made on a case-by-case basis by the instructor
and typically are made only for
documented health or university-sponsored activities.

\emph{Show your work.} Include an appropriate amount of detail:
(1) When you use a formula, always show the general formula (with
    math symbols) before plugging in actual numbers.
(2) After you have plugged in numbers, be concise with routine arithmetic.
(3) Include important middle steps, formulas, intermediate quantities.
\emph{Lack of important middle steps and statements
will cost you partial or whole credit.}
Study and follow examples in the textbook and homework solutions
in terms of sequence of steps,
amount of calculation details,
verbal explanations, and conclusion.

All homework assignments may not be worth the same points due to their
difference in work-load and difficulty.

\emph{Presentation counts.}
You should \emph{work out the problems first on scratch paper},
then write down your answers in a clean, organized way.
\emph{Messy, incomplete, or unclear presentation},
as determined by the grader, will cost you partial credit.
\emph{You are strongly recommended to prepare your final answer on a computer and print it
out.}
In order to encourage this preferred and professional practice,
each homework produced on the computer will earn you 1 extra point
(on the scale of the perfect overall grade, 100, for the course).
%The preferred tool for mathematical writing is \LaTeX.
%\LaTeX source files of some course materials are posed on Blackboard for
%your reference.

\emph{Graphics must be prepared on a computer.}
Hand-made graphics will not be graded.

\emph{Regular attendance is necessary} for success in this course.
To help enforce your attendance,
you get 5 attendance points by default.
Every time you are found to be absent, you lose 1 point (until you lost
all 5 points).
(I occasionally, and randomly, ask particular students to answer
questions, and you better respond on site!)

Your final grade will be calculated based on the following proportions:


\hskip2cm
\begin{tabular}{ll}
Homework        & 40\%\\
Midterm exam   & 20\%\\
Final exam      & 35\%\\
Attendance      & 5\%\\
Presentation    & up to 6\% extra
\end{tabular}

Conversion used to determine a letter grade:

\[
\overset{0}{|}
\text{------F------}\overset{50}{|}
\text{------D------}\overset{65}{|}
\text{------C------}\overset{80}{|}
\text{------B------}\overset{90}{|}
\text{------A------}\overset{100}{|}
\qquad\qquad\qquad\strut
\]

\emph{The final grades will NOT be curved.}

According to
the ``Grading System and Grade Point Average Computation''
in the University Catalog,
`A' is an `honor grade',
`B' indicates `outstanding performance',
`C' indicates `average performance',
`D' (the lowest passing grade) indicates `below average performance',
and
`F' indicates `failure'.

Excerpt of University grading policy regarding `C' and `D':
\begin{verbatim}
*Implications of the Grade of ‘C’ (and below) for letter-graded undergraduate courses which are:
--*Prerequisites for other courses, or*
--*Degree major requirements, or*
--*Core courses*

*
*
C+ (2.3): Satisfactory to Fair: satisfactory level of performance, with some mastery of material.

C (2.0): Average: satisfactory level of performance and level of competency in the subject.
A minimum grade of 'C' (2.0) is required for all prerequisites and major courses.

C- (1.7): Barely satisfactory: Minimum grade required for all Core (X) Courses.
A grade of C- (1.7) in a class which is a prerequisite for another class or in a class required
for a student's major will result in the student being required to retake the class.

D+ (1.3); D (1.0); D- (0.7): Below Average: Fair to poor level of competency in the subject matter.
A grade of D+, D or D- in a Core (X) class will automatically require the student to retake
the class to receive core credit, starting Fall 2011.
\end{verbatim}

The University Catalog 2011--2012,
at \url{www.uaf.edu/catalog/catalog_11-12/academics/index.html},
and policies of the Department of Math and Stats,
at \url{www.uaf.edu/dms/policies/},
contain further information regarding
grading and exams.

\bigskip
\textbf{\large Ethics}:
Studying together, and getting study assistance from a tutor, is allowed
and encouraged. Copying someone else's work and representing it as your
own is plagiarism. Plagiarism and other forms of cheating in homework
and exams may result in a 0 (zero) score for the homework/exam
involved.

Please read ``Student Code of Conduct'' in the University Catalog
2011--2012
at
\url{www.uaf.edu/catalog/catalog_11-12/academics/regs3.html#Student_Conduct}.


\bigskip
\textbf{\large Withdrawal:}
I may withdraw any student from class who
(1) misses midterm exam 1 without a valid reason OR
(2) misses two homework assignments (by the deadline for withdrawal,
i.e.\@ May 23).


\bigskip
\textbf{\large Disability Services}:
If you have a physical handicap or learning disability, please make me
and the Office of Disabilities Services (474-7043) aware of the
situation so that reasonable accommodations can be made.

\bigskip
\textbf{\large Tentative schedule}: see separate page.


\end{document}
