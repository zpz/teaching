\documentclass[12pt]{article}
\usepackage{coursenote}
\begin{document}
\title{STAT 300 Midterm 1 Review Topics}
\maketitle

Review: lecture notes, examples and exercises mentioned in lecture
notes, textbook, homework problems, additional problems in the ``problem
book''.

Bring 4 sheets (8 sides) of notes, prepared by yourself and
a calculator.

Statistical tables (for the standard normal cdf) will be provided.


\section{Concepts, definitions}

\begin{itemize}
\item Sample space, events.
\item Conditional, independence.
\item The transition from ``experiments'', ``outcomes'', ``events'' to
    ``random variables'' and ``distributions''.
\item For discrete and continuous, univariate and bivariate:
    pmf, pdf, cdf, expectation (aka expected value or mean).
\item Quantiles.
\item For bivariate: marginal, conditional, independence.
\item For univariate: variance, standard deviation.
\item For bivariate: covariance, correlation.
\end{itemize}

\section{Computations}

\begin{itemize}
\item Use properties of probability, product rule,
    combination/permutation, conditional/independence, etc.
    to calculate probabilities.
\item Calculate probabilities about a normal distribution:
    standard, nonstandard, standardization, table lookup,
    ``critical value $z_{\alpha}$''.
\item Calculate probabilities given a discrete distribution:
    summation.
\item Calculate probabilities given a continuous distribution:
    integration.
\item In integrating a joint pdf, be sure to get the limits right when
    the domain is not rectangular.
\item Univariate: from pdf/pmf to cdf; from cdf to pdf/pmf.
\item Bivariate: from joint to marginal and conditional.
\item Be able to tell (via calculations) whether the two components in a
    (discrete or continuous) bivariate distribution are independent.
\item Compute mean, variance, covariance, correlation given (joint) distribution.
\item Compute mean and variance of a linear combination of other random
variables.
\end{itemize}

\section{Specific facts, formulas}
\begin{itemize}
\item Sample mean, sample variance, sample standard deviation.
\item Formulas for combination (``choose''), permutation.
\item The ``computational formula'' connecting (co)variance and mean.
    (Versions for sample, univariate population, and bivariate
    population.)
\item Binomial: binomial experiment; you should be able to tell whether
    a binomial distribution applies given the description of the problem;
    pmf, mean, variance.
\item Continuous uniform: pdf, cdf, mean, variance.
\item Normal: pdf, mean, variance; you should be able to tell a pdf is
    normal, and figure out its mean and variance from the formula.
\item Exponential: pdf, cdf, ``memoryless'' property.
\end{itemize}

\section{Plots}

Histogram\\
boxplots, comparative boxplots\\
normal probability plots (i.e.\@ Q-Q plots):

\indentblock{%
Be able to create these plots;
be able to make some comments and interpretation on the plots
(in other words, know what each type of plots is used for);
histogram on ``density scale''.}

This exam won't test on these plots.

Venn diagram

\end{document}
