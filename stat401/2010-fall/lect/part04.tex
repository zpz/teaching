\documentclass[12pt]{article}
\usepackage{coursenote}

\begin{document}
\title{STAT 401 Chapter 5.1--5.7}
\maketitle


A {\bf matrix} is a rectangular array of numbers, arranged in rows and columns.
Matrices are used extensively in math and statistics.  They provide a convenient
way to express linear statistical models, as we shall see in the later sections
of chapter~5.
The matrix formulation for simple linear regression (linear regression with
just one predictor) is identical to the matrix formulation for
multiple linear regression (regression using two or more predictors).
This makes the transition from simple linear regression (SLR) to
multiple linear regression (MLR) smooth and seamless.  Formulas for SLR
work for MLR as well.


Matrix notation:\\
(1) conventional bracket notation;\\
(2) upper-case letter notation; (sometimes boldface; but we don't
enforce it)\\
(3) element listing: $\mat{A} = [a_{ij}]$,
$i = 1,\dotsc,m$, $j = 1,\dotsc, n$;\\
(4) dimension (size): $m \times n$, $m$ by $n$;
(5) column vectors and row vectors: often denoted by
boldface lower-case letters.

\section{Matrices}

Here are several matrices that I'll use as examples:

\[
 A =  \left[ \begin{array}{cc}  1 & 2 \\ 3 & 4  \end{array} \right],
 \hspace{.2in}
 B =  \left[ \begin{array}{ccc}  3 & -1 & 2 \\ 2 & 0 & 1  \end{array} \right],
 \hspace{.2in}
 C =  \left[ \begin{array}{ccc}  4 & 7 & -2   \end{array} \right],
 \hspace{.2in}
 D =  \left[ \begin{array}{c}  1 \\ 2 \\ 3 \end{array} \right],
 \]
 \[
 E =  \left[ \begin{array}{cc}  2 & -1 \\ 1 & 3 \end{array} \right],
 \hspace{.2in}
 F =  \left[ \begin{array}{c}  -1 \\ 1 \end{array} \right],
 \hspace{.2in}
 G =  \left[ \begin{array}{cc}  1 & 3 \\ 3 & 5 \end{array} \right],
 \hspace{.2in}
 H =  \left[ \begin{array}{ccc}  1 & 3 & 5 \\ 2 & 6 & 10  \end{array} \right].
 \]

The {\bf dimension} or {\bf size} of a matrix
is described by the number of rows and the number
of columns, separated with the word ``by'' or the symbol ``$\times$''.  For example,
the dimensions of $A$, $B$, $C$, and $D$ are:

$$ 2 \times 2, 2 \times 3, 1 \times 3, {\mbox{ and }} 3 \times 1. $$

The number of rows is always listed before the number of columns.

A {\bf square matrix} is one which has the same number of rows and columns.
Examples: $A$, $E$ and $G$ are square matrices.

A {\bf column vector} is a matrix with just one column, and a {\bf row vector}
is a matrix with just one row. Examples: $C$ is a row vector; $D$ and
$F$ are column vectors.

The {\bf transpose} of a matrix is a matrix in which the rows and columns
are interchanged, and it is denoted by either a `prime' symbol (as in $A'$) or
a superscript $T$ or a superscript $t$ (as in $A^T$ or $A^t$).  Examples:

$$ A' =  \left[ \begin{array}{cc}  1 & 3 \\ 2 & 4  \end{array} \right],
 B' =  \left[ \begin{array}{cc}  3 & 2 \\ -1 & 0 \\ 2 & 1  \end{array} \right],
 {\mbox{ and }}
C' =  \left[ \begin{array}{c}  4 \\ 7 \\ -2   \end{array} \right]$$

Two matrices are said to be equal if they have the same dimensions and the entries
in one are identical to the entries in the other.

A {\bf symmetric matrix} is a square matrix which is equal to its transpose.
Example: $G$ is a symmetric matrix; $A$ and $E$ are not.


\section{Matrix addition and subtraction}

If two matrices have the same dimensions, it is possible to
add (subtract) them; the resulting matrix is the same size, and the entries of the
result are obtained by adding (subtracting, respectively) corresponding
entries.

For the matrices listed above, $A$, $E$ and $G$ can be added to
or subtracted from one another; $B$ and $H$ can be added to or subtracted from
one another.  Examples:
$$ A + E = 
    \left[ \begin{array}{cc}  1 & 2 \\ 3 & 4  \end{array} \right]
 +  \left[ \begin{array}{cc}  2 & -1 \\ 1 & 3 \end{array} \right]
 =  \left[ \begin{array}{cc}  1+2 & 2-1 \\ 3+1 & 4+3 \end{array} \right]
 =  \left[ \begin{array}{cc}  3 & 1 \\ 4 & 7 \end{array} \right]$$

$$ A - E = 
    \left[ \begin{array}{cc}  1 & 2 \\ 3 & 4  \end{array} \right]
 -  \left[ \begin{array}{cc}  2 & -1 \\ 1 & 3 \end{array} \right]
 =  \left[ \begin{array}{cc}  1-2 & 2-(-1) \\ 3-1 & 4-3 \end{array} \right]
 =  \left[ \begin{array}{cc}  -1 & 3 \\ 2 & 1 \end{array} \right]$$


\section{Matrix multiplication}

a) Multiplying a matrix by a scalar:

The word `scalar' is a technical term for a number.  When multiplying a 
matrix by a scalar, simply multiply all the entries in the matrix by the
scalar.  The scalar is always written to the left of the matrix.

For example,
$$5A = 
    5\left[ \begin{array}{cc}  1 & 2 \\ 3 & 4  \end{array} \right]
 =  \left[ \begin{array}{cc}  (5)(1) & (5)(2) \\ (5)(3) & (5)(4)  \end{array} \right]
 =  \left[ \begin{array}{cc}  5 & 10 \\ 15 & 20 \end{array} \right]$$

$$-2B =  -2\left[ \begin{array}{ccc}  3 & -1 & 2 \\ 2 & 0 & 1  \end{array} \right]
 = \left[ \begin{array}{ccc}  -6 & 2 & -4 \\ -4 & 0 & -2  \end{array} \right]$$

b) Multiplying a row vector by a column vector (where the row vector is written
first and the column vector is written second):
This operation is defined when (and only
when) the vectors have the same number of entries.
If the product is defined, its value is given
by summing the products of the entries in the two vectors.

\bigskip

Example:  We note that $C$ is a row vector, and $D$ and $F$ are column vectors.
The product $CD$ is defined (has meaning) because both $C$ and $D$ have 3 entries;
the product $CF$ is not defined.  We have:
$$ CD =  \left[ \begin{array}{ccc}  4 & 7 & -2   \end{array} \right]
  \left[ \begin{array}{c}  1 \\ 2 \\ 3 \end{array} \right]
  = (4)(1) + (7)(2) + (-2)(3) = 12.$$

c) Multiplying one matrix by another:  Some products are possible, others are not.
Whether a product is defined or not depends on the sizes of the two matrices.

You might guess at this point that matrix multiplication is much like
matrix addition and subtraction, and that we could only multiply two
matrices if they had the same dimension.  For example you might think that

$$ 
    \left[ \begin{array}{ccc}  1 & 2 & 3 \\ 4 & 5 & 6  \end{array} \right]
     *
    \left[ \begin{array}{ccc}  2 & -1 & 0 \\ 1 & 3 & 5  \end{array} \right]
    =
    \left[ \begin{array}{ccc}  (1)(2) & (2)(-1) & (3)(0) \\ (4)(1) & (5)(3) & (6)(5)  \end{array} \right]
    =
    \left[ \begin{array}{ccc}  2 & -2 & 0 \\ 4 & 15 & 30  \end{array} \right]
$$

This type of multiplication exists.  This is called the `Hadamard product.'
It is used by a small number of  mathematicians.  We do not use it.

Instead, we use the standard matrix multiplication, which may initially seem
somewhat contrived and artificial; but it actually has some marvellous
mathematical properties, as we shall see.

If $A$ is an $m \times n$ matrix and $B$ is a $p \times q$ matrix, then:

  \begin{enumerate}
  \item The product $AB$ is defined if (and only if) $n = p$.
  \item If $n=p$, then:
    \begin{enumerate}
    \item The product $AB$ is defined and its size is $m \times q$.
    \item The entry in the $r^{th}$ row and $c^{th}$ column of $AB$ is
    the product of the $r^{th}$ row of $A$ and the $c^{th}$ column of $B$
    (as defined in (b), at the bottom of the previous page).
    \end{enumerate}
  \end{enumerate}

Note: If $n=p$ so that the product $AB$ is defined, the matrices $A$ and $B$ are
said to be `conformable.'

Here's an easy way to recognize a pair of conformable matrices: Write down their
dimensions in a row; this is a list of 4 numbers.  If the inner pair of numbers
are the same, then the matrices are conformable; and in this event, the outer pair of
numbers gives the size of the resulting product matrix.

Example:  From before,
\[
  B =  \left[ \begin{array}{ccc}  3 & -1 & 2 \\ 2 & 0 & 1  \end{array} \right]
 \hspace{.1in}
 {\mbox{ and }}
 \hspace{.1in}
 D =  \left[ \begin{array}{c}  1 \\ 2 \\ 3 \end{array} \right].
 \]

$B$ is $2 \times 3$ and $D$ is $3 \times 1$.  So we write:
$$(2 \times 3) (3 \times 1).$$  The inner pair of numbers are the same (both are 3),
so the product is defined.  And the resulting matrix, $BD$, is $2 \times 1$.

The entry in the (1,1) position (first row, first column) of $BD$ is:
$$ \left[ \begin{array}{ccc}  3 & -1 & 2 \end{array} \right]
 \left[ \begin{array}{c}  1 \\ 2 \\ 3 \end{array} \right]
 = (3)(1) + (-1)(2) + (2)(3) = 7.$$

The entry in the (2,1) position (second row, first column) of $BD$ is:
$$ \left[ \begin{array}{ccc}  2 & 0 & 1 \end{array} \right]
 \left[ \begin{array}{c}  1 \\ 2 \\ 3 \end{array} \right]
 = (2)(1) + (0)(2) + (1)(3) = 5.$$

Therefore $BD = \left[ \begin{array}{c}  7 \\ 5 \end{array} \right].$

d) Dividing one matrix by another:  There is no such thing as matrix division, 
so this is a very short section.

Usually
\[
\mat{A}\mat{B} \ne \mat{B}\mat{A}
.
\]
Unless both $\mat{A}$ and $\mat{B}$ are square
(and of the same size), the two sides are not both defined!
When both sides are defined, they usually are still unequal.

\[
(\mat{A} \mat{B}) \mat{C}
= \mat{A} (\mat{B} \mat{C})
\]
if the sizes are conformable.
Because of this property, we can write
$\mat{A}\mat{B}\mat{C}$
and the result is not ambiguous.

\[
\mat{C}(\mat{A} + \mat{B})
= \mat{C}\mat{A} + \mat{C}\mat{B}
\]



\section{Special types of matrices}

{\bf Symmetric matrices:}
A matrix $A$ is {\bf{symmetric}} if it is equal to its transpose: $A = A^T$.
Pictorially, if you draw a line through the matrix from the upper left corner to the
lower right corner, the lower portion of the matrix is the mirror image of the
upper portion of the matrix.

Example. This matrix is symmetric:
$$A = \left[ \begin{array}{ccc}  3 & 2 & 1 \\ 2 & 0 & -1 \\ 1 & -1 & 0 \end{array} \right]$$

$\bullet$ Fact: All symmetric matrices are square (that is, the number of rows is equal
to the number of columns).

{\bf Diagonal matrices:}
A matrix is {\bf{diagonal}} if it is square and all its off-diagonal entries are zero.

Examples. Both of the following matrices are diagonal:

$$A = \left[ \begin{array}{ccc}  3 & 0 & 0 \\ 0 & 1 & 0 \\ 0 & 0 & 3 \end{array} \right]
{\mbox{ and }}
I = \left[ \begin{array}{cccc}  1 & 0 & 0 & 0 \\
 0 & 1 & 0 & 0 \\ 0 & 0 & 1 & 0 \\ 0 & 0 & 0 & 1 \end{array} \right]$$

{\bf{Identity matrix:}}
A diagonal matrix whose main-diagonal entries are all 1 is called an {\bf{identity matrix}}.
We use `$I$' to denote an identity matrix.  (The same letter is used, whether the
matrix is $2\times 2$ or $3 \times 3$, etc.)

Fact: If $A$ is an $n \times n$ square matrix, and $I$ is the $n \times n$ identity
matrix, then $AI = IA = A.$

The identity matrix $I$ plays the same role for matrices that the number 1 does for
real numbers: Multiplying a number by 1 (on the left or right) leaves that number
unchanged.  Multiplying a square matrix by $I$ (on the left or right) leaves that
matrix unchanged.

Vector of 1's: $\mat{1}$

Matrix of 1's: $\mat{J}$ (page 187).\\
(This does not appear to have a standard notation.
Encountered less often.)

All zero's: $\mat{0}$.\\
(In addition and subtraction, $\mat{0}$ changes nothing.
In multiplication, $\mat{0}$ destroys everything;
but it may say something about the size of the resultant matrix.)


\section{Inverse of a matrix}

Given \emph{square} matrix $\mat{A}$,
if there exists $\mat{B}$ such that
\[
\mat{A}\mat{B} = \mat{I}
\]
then $\mat{A}$ and $\mat{B}$ are the ``inverse (matrix)'' of each
other.
We say ``invert'' a matrix, or ``take the inverse''.
We use $\mat{A}^{-1}$ to denote the inverse of $\mat{A}$.

If someone asks whether two particular square matrices are inverses of one another,
the easiest thing to do is to multiply them together (in either order).
If the product is the identity matrix, then the original two matrices are
inverses of one another; otherwise they are not inverses.

For a $2 \times 2$ matrix $A$, there is a simple formula for the inverse.
This formula is valid if the determinant,
$\operatorname{det}(\mat{A})$,
is non-zero (otherwise $A$ has no inverse):

\[ {\mbox{If }} 
A =  \left[ \begin{array}{cc}  a & b \\ c & d  \end{array} \right],
{\mbox{ then }}
A^{-1} = \frac{1}{ad-bc}
 \left[ \begin{array}{cc}  d & -b \\ -c & a  \end{array} \right]
 = \frac{1}{{\mbox{det}}(A)}
 \left[ \begin{array}{cc}  d & -b \\ -c & a  \end{array} \right].\]

\example
If 
$A = \left[ \begin{array}{cc}  2 & 1 \\ 4 & -3  \end{array} \right],$
 then det($A$) = (2)(-3)-(1)(4) = -10,
 so $$A^{-1} = \frac{1}{-10} 
  \left[ \begin{array}{cc}  -3 & -1 \\ -4 & 2  \end{array} \right]
  =  \left[ \begin{array}{cc}  3/10 & 1/10 \\ 4/10 & -2/10 \end{array} \right].$$

\bigskip

The inverse of a matrix can be used to solve certain equations involving
matrices.  If we have a matrix equation such as $$A\mathbf{x} = \mathbf{b},$$
where $A$ is $n \times n$ and $\mathbf{x}$ is a $n \times 1$ column vector of unknowns
and $\mathbf{b}$ is a $n \times 1$ column vector, then we can solve for
$\mathbf{x}$
by multiplying both sides on the left by $A^{-1}$, like so:
$$A^{-1}A\bf{x} = A^{-1}\mathbf{b}.$$
But $A^{-1}A\mathbf{x} = I\mathbf{x} = \mathbf{x},$ so $$\mathbf{x} = A^{-1}\mathbf{b}.$$

\example
Solve the following matrix equation:
$$\left[ \begin{array}{cc}  0 & 3 \\ 1 & -1  \end{array} \right]
 \left[ \begin{array}{c} x_1 \\ x_2 \end{array} \right]
= \left[ \begin{array}{c} 1 \\ 2 \end{array} \right].$$

Solution: 
Let $$A = \left[ \begin{array}{cc}  0 & 3 \\ 1 & -1  \end{array} \right].$$
Using the formula for the inverse of a $2\times 2$ matrix,
$$ A^{-1} = \frac{1}{(0)(-1)-(3)(1)}
\left[ \begin{array}{cc}  -1 & -3 \\ -1 & 0  \end{array} \right]
= \frac{1}{-3}\left[ \begin{array}{cc}  -1 & -3 \\ -1 & 0   \end{array} \right]
= \left[ \begin{array}{cc}  1/3 & 1 \\ 1/3 & 0  \end{array} \right].$$
The matrix equation can then be written as follows:
$$A\left[ \begin{array}{c} x_1 \\ x_2 \end{array} \right] = \left[ \begin{array}{c} 1 \\ 2 \end{array} \right].$$
Then $$A^{-1}A\left[ \begin{array}{c} x_1 \\ x_2 \end{array} \right] = A^{-1}
\left[ \begin{array}{c} 1 \\ 2 \end{array} \right].$$
Therefore
$$I\left[ \begin{array}{c} x_1 \\ x_2 \end{array} \right] = A^{-1}\left[ \begin{array}{c} 1 \\ 2 \end{array} \right].$$
Therefore
$$\left[ \begin{array}{c} x_1 \\ x_2 \end{array} \right] = A^{-1}\left[ \begin{array}{c} 1 \\ 2 \end{array} \right]
= \left[ \begin{array}{cc}  1/3 & 1 \\ 1/3 & 0  \end{array} \right]\left[ \begin{array}{c} 1 \\ 2 \end{array} \right]
= \left[ \begin{array}{cc} (1/3)(1)+(1)(2) \\ (1/3)(1)+(0)(2)  \end{array} \right]
= \left[ \begin{array}{cc} 7/3 \\ 1/3  \end{array} \right].$$


\alert[Proposition]%
If $\mat{A}$ has inverse $\mat{B}$, then\\
(1) $\mat{B}$ is unique.\\
(2) $\mat{A}\mat{B} = \mat{B}\mat{A} = \mat{I}$.\\
(3) $\mat{A}^T$ has inverse $\mat{B}^T$.

To show property (3), one only needs to show
$\mat{A}^T \mat{B}^T = \mat{I}$.
This is apparent since
$\mat{A}^T\mat{B}^T = (\mat{BA})^T = \mat{I}^T = \mat{I}$.

\alert[Proposition]%
If $\mat{A}$ and $\mat{B}$ both have inverses,
then $(\mat{A}\mat{B})^{-1} = \mat{B}^{-1}\mat{A}^{-1}$.

To show this, notice
\[
(\mat{AB}) \bigl(\mat{B}^{-1}\mat{A}^{-1}\bigr)
= \mat{A}\bigl(\mat{B}\mat{B}^{-1}\bigr)\mat{A}^{-1}
= \mat{A}\mat{I}\mat{A}^{-1}
= \mat{A}\mat{A}^{-1}
= \mat{I}
\]

Note: $(\mat{A} + \mat{B})^{-1}$ is not $\mat{A}^{-1} + \mat{B}^{-1}$.
(The sum may not even have an inverse.)

\textbf{Recap of basic points:}
\begin{enumerate}
\item We speak of the inverse of \emph{square} matrices only.
\item If the inverse exists, it is unique.
\item A square matrix may not have an inverse (may not be invertible).
\end{enumerate}

The inverse exists iif (if and only if)
$\mat{A}$ is non-singular,
and this is equivalent to saying either
(1) the determinant of $\mat{A}$ is nonzero,
or
(2) $\mat{A}$ has full rank
(these two conditions are equivalent).


\section{Rank and determinant}

\subsection{Rank}

Section 5.5 introduces the notions of `linear independence' (of the rows or
columns of a matrix) and `rank' (of a matrix) in order to provide a basis for
Section 5.6, which discusses the concept of the `inverse of a (square) matrix.'

Rank is a property of any matrix.
But here we worry about \emph{square} matrix only.

The ``rank'' of $\mat{A}$ is the number of
\emph{linearly independent} columns (or rows) of it.
(Page 188.)

We say ``(or rows)'' because examining columns and rows will give you
the same number.

If the rank of $\mat{A}_{n\times n}$ is $n$,
we say it has ``full rank'' and is ``non-singular''.
\emph{In order for a matrix to have an inverse
(discussed at some length in Section 5.6),
the matrix must be square and it must be ``of full rank.''
It then has a non-zero determinant, and is invertible.
If a matrix is not `of full rank,' then it does not have an inverse.}

\definition Linear indepdendence, page 188.

You need to know this
definition and use it to determin whether two or three simple vectors
are linearly independent.

To show dependence, you only need to come up with one counter example.
To show independence, you need to prove it---the coefficients
must be all zero in order for the linear combination to be the zero
vector.

We care about whether a square matrix if of full rank.
Its actual number (if not $n$) rarely enters calculations.
This is different from $\det{(\mat{A})}$,
which appears in the density function of normal distribution,
which is, of course, central to statistics.

\subsection{Determinant}

Each \emph{square} matrix has a corresponding, unique number, called its
``determinant'', $\det{(\mat{A})}$, or sometimes $|\mat{A}|$.
(The second notation should be used only when we know this number
for the matrix in question is guaranteed to be non-negative.
This is an important property and some matrices have this property.
A type of such matrices (covariance matrices) are very important in statistics.)

The definition of determinant is essentially how to calculate this
number. Following the formula, a square matrix results in a unique
number. We will see how to calculate the determinant of
a $2 \times 2$ matrix;
but don't worry about how to calculate this number in general.

Because the number is unique,
it's a ``property'' of a matrix, or a ``function'' (from a matrix to a
real value) of a matrix.
In this sense I can call it an ``operation'' on the matrix.

If $\det{(\mat{A})} \ne 0$, the matrix is ``good'' (for us):
it is non-singular and it is invertible.

If det($A$) = 0, we say that $A$ is a `singular' matrix; in this case, $A$ does
not have an inverse. (``$A$ is non-invertible.'')

For a $2 \times 2$ matrix 
$$A = \left[ \begin{array}{cc}  a & b \\ c & d  \end{array} \right],$$
det($A$) is defined by:
$${\mbox{det}}(A) = ad - bc.$$

% For a $3 \times 3$ matrix 
% $$A = \left[ \begin{array}{ccc}  a & b & c \\ d & e & f \\ g & h & k  \end{array} \right],$$
% det($A$) is defined by:
% $${\mbox{det}}(A) = a(ek-fh)-b(dk-fg)+c(dh-eg).$$


\section{Some basic results for matrices}

\begin{itemize}
\item $A + B = B + A$.  (Matrix addition is commutative -- order does not matter.) (5.25)
\item $(A+B)+C = A+(B+C)$. (Matrix addition is associative.) (5.26)
\item $(AB)C = A(BC)$. (Matrix multiplication is associative.) (5.27)
\item $C(A+B) = CA + CB$. (The distributive property for matrices.) (5.28)
\item $k(A+B) = kA + kB.$ (Here, $k$ is a scalar. This law says that we can distribute a scalar.) (5.29)
\item $(A')' = A$ (or $(A^T)^T = A$.) (If you take the transpose twice,
 you get back the original matrix.) (5.30)
\item $(A+B)^T = A^T + B^T$. (The transpose of a sum is equal to the sum of the transposes.) (5.31)
\item $(AB)^T = B^T A^T$. (The transpose of a product is equal to the product of the transposes --
written in the reverse order.) (5.32)
\item $(ABC)^T = C^T B^T A^T$ (Says that (5.32) extends to 3 (or more) matrices.) (5.33)
\item $(AB)^{-1} = B^{-1}A^{-1}$. (The inverse of a product is equal to the product of the inverses --
written in the reverse order.  $A$ and $B$ must both be square, and both must have an inverse!) (5.34)
\item $(ABC)^{-1} = C^{-1}B^{-1}A^{-1}$ (Says that 5.34 extends to 3 (or more) matrices.) (5.35)
\item $(A^{-1})^{-1} = A.$ (If you take the inverse of a matrix two times, you get back the
original matrix -- provided $A$ has an inverse to begin with.) (5.36)
\item $(A^T)^{-1} = (A^{-1})^T$.  (Whether you take the inverse first or the transpose first
does not matter; you get the same answer both ways.  $A$ must be an invertible matrix.) (5.37)
\end{itemize}

To show the last property, notice
\[
\mat{A}^{T} \bigl(\mat{A}^{-1}\bigr)^T
= \bigl(\mat{A}^{-1}\mat{A}\bigr)^T
= \mat{I}^T
= \mat{I}
\]

\end{document}
