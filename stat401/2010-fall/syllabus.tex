\documentclass{article}
\usepackage{techart}
\parindent=0pt
\parskip=5pt

\begin{document}
\title{STAT F401: Regression and Analysis of Variance, Fall 2010}
\author{%
Lectures: MWF 11:45--12:45 Chapman 106\\
Labs: M 2:15--5:15 (F02), Tu 9:40--12:40 (F01) Bunnell 407\\
Final exam: 10:15--12:15, Wed Dec.~15 Bunnell 407}
\date{}
\maketitle

\textbf{\large Instructor: Zepu Zhang.}

zzhang6@alaska.edu;
306A Chapman, 474-7605.
Mail box: Chapman 101.

Office Hours: MWF 2:30--3:30,
or by appointment.

\bigskip
\textbf{\large Teaching Assistant: Jason Waite.}

jason.waite@alaska.edu;
303 Chapman, 474-6174;
108 O'Neill, 474-7839.
Mail box: Chapman 101.

Office hours: Tu 1:30--4:30,
or by appointment.

\bigskip
\textbf{\large Textbook}:

Required:
\begin{itemize}
\item
\textit{Applied Linear Statistical Models}, 5th edition, by Michael~Kutner,
Christopher~Nachtsheim, John~Neter, and William~Li.  McGraw-Hill/Irwin,
2005.
\item
\textit{An Introduction to R}, available at
\texttt{http://www.r-project.org}.
\end{itemize}

\bigskip
\textbf{\large Blackboard}:
Use the Blackboard site for this course to access
schedule, announcements, lecture notes, homework assignments and
solutions, grades, and other related materials.
Some documents may see small modifications/updates after first posting.

\bigskip
\textbf{\large Prerequisites:}
Stat 200, Stat 300 or an equivalent course is necessary preparation; a grade of B
or better in the previous course is recommended.  If you have not had a statistics
course, or if it has been a long time since you've had a statistics course or used
statistics, you are strongly advised to take Stat 200 or Stat 300 before taking
Stat 401.  Appendix A in the KNNL text may be useful for reviewing.

\bigskip
\textbf{\large Goals and expected learning outcomes:}
\begin{itemize}
\item Learn simple and multiple regression, including multiple and
partial correlation, the extra sum of squares principle, indicator variables,
and model selection techniques.
\item Learn how to conduct one-way analysis of variance for
completely randomized designs.
\item Learn how to use graphics and specific tests to determine if the 
modeling assumptions are met.
\item Learn to use \texttt{R} for general computation, graphics, and
simple programming in the context of fitting and assessing linear
models.
\end{itemize}

\bigskip
\textbf{\large Computation and software:}

We will use the free software \texttt{R}.
Every student should download it from
\texttt{http://www.r-project.org} and install it on their own computer.
If \texttt{R} is whole new software to you,
the available classroom time will NOT be enough to get you really
comfortable with it.
Expect to spend time learning and exploring \texttt{R} on your own.
However, the instructor and TA will be able to help you in learning
\texttt{R}.

Although the course treats \texttt{R}
more as a tool than as a topic in itself,
no previous experience with \texttt{R} is assumed.
Previous experience in computer programming will be very helpful.

\textbf{Note on learning \texttt{R}}:
all \texttt{R} handouts (such as lab handouts) are supposed to
be \emph{tried}, not just \emph{read}.


\bigskip
\textbf{\large Lab and homework:}

The computer labs are designed for you to\\
(1) learn to use \texttt{R};\\
(2) learn to use \texttt{R} to conduct computations pertaining to
the lectures' content;\\
(3) do homework problems and get help from the TA.

The labs are required regardless of whether a particular lab session
assigns a write-up to submit.

Each homework assignment consists of questions, problems,
and possibly write-up from the most recent lab session.
Exact content of the lab write-up, if any,
will be announced.

Homework should be typeset/word-processed (preferred) or hand-written neatly.
Turn in hard copies only. Remember to number the pages and use a stapler.
Homework should be submitted to the instructor or the TA
in class or in mailbox.

Show middle steps, formulas, main \texttt{R} code and output,
for each homework problem.

Special note about computer output in homework submission:\\
(1) Never print the computer output directly; instead, include it in
your write-up and perform necessary editing, omission, decoration as
explained below.\\
(2) Only include sections of the output that show significant steps, final
results, etc. Never list whole datasets; instead, keep the first few and
last few numbers and use ``...'' to signal omission.\\
(3) Do not let one graph consume a whole page unless the amount of
information in the graph justifies this size. Usually the graph should
be resized and integrated into the document.\\
(4) Use circles and arrows to mark out important numbers in the computer output.

\bigskip
\textbf{\large Lecture notes:}
Lecture notes serve as outlines and pointers.
They are not meant to be a polished reader.
Unless otherwise announced,
the notes mention all required topics,
and textbook topics not mentioned in the notes are less important.
It may happen that certain content of the notes
do not get enough discussion time in the lectures;
that content is still required material.

Lab handouts should be treated similarly.

\bigskip
\textbf{\large Policy on late homework:}
Late homework will result in the loss of 20\% of the points each additional
(business) day that it is late (weekend counts as one business day);
thus homework that is more than one full week late will receive no credit.
Exceptions are made
on a case-by-case basis by the instructor and typically are made only for
documented health or university-sponsored activies.


\bigskip
\textbf{\large Grading policy:}

There will be two in-class hour-long midterm exams
and one two-hour final exam.
Coverage of each exam will be announced later;
the final exam will in principal cover the entire course.

The exams will be closed-book. You may use notes \emph{prepared by
yourself} on
two sheets of $8 \frac{1}{2} \times 11$ inch paper (both sides).
You may not use a computer in the exams.
You may use a calculator.
(It must be a calculator only; it must not be a device that can store
notes.)

In all homework and exams,
messy presentation and incomplete/unclear writing, as determined by the
grader, may cost partial credit.

In all homework and exams,
intermediate steps that show your understanding
of the topic and procedure are as important as the final answer.
Both the procedure and the final answer carry credit.

Your final grade will be calculated based on the following proportions:

\hskip2cm
\begin{tabular}{ll}
Homework (incl labs) & 40\%\\
Midterms & 30\% ($15\% \times 2$)\\
Final & 30\%
\end{tabular}

Grading scale:
A (honor grade): 90--100;
B (outstanding): 80--89.99;
C (average): 70--79.99;
D (below average): 60--69.99;
F (failure): 0--59.99.

\bigskip
\textbf{\large Ethics}:
Studying together, and getting study assistance from a tutor, is allowed
and encouraged. Copying someone else's work and representing it as your
own is plagiarism. Plagiarism and other forms of cheating in homework
and exams may result in a 0 (zero) score for the homework/exam
involved.

Please read ``Student Code of Conduct'' on pages~117--118 of the
\emph{Class Schedule}, and policies of the Department of Math and Stats
at www.dms.uaf.edu/dms/Policies.html.


\bigskip
\textbf{\large Disability Services}:
If you have a physical handicap or learning disability, please make me
and the Office of Disabilities Services (474-7043) aware of the
situation so that reasonable accommodations can be made.


\bigskip
\textbf{\large Withdrawal:}
I may withdraw any student from class who
(1) misses an exam without a valid reason OR
(2) misses three homework assignments.

\end{document}
