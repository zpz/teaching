\documentclass{article}
\usepackage{techart}
\parindent=0pt
\parskip=5pt

\begin{document}
\title{STAT F611: Time Series, Spring 2011}
\author{MWF 1:00--2:00 Gruening 309}
\date{}
\maketitle

\textbf{\large Instructor: Zepu Zhang.}
306A Chapman, zzhang6@alaska.edu, 474-7605.\\
Mail box: Chapman 101.

Office Hours: MWF 3:30--5:00; or by appointment.

\bigskip
\textbf{\large Textbook}:
\textit{Time Series Analysis With Applications in R},
by Jonathan D.\@ Cryer and Kung-Sik Chan,
2nd ed, 2008, Springer.
We plan to cover chapters 1--11 and possibly 13.

Full-text of this book is available online at
www.springerlink.com (only if you access it on the UAF campus, I
suppose).

\bigskip
\textbf{\large Blackboard}:
A course Blackboard site has been set up. You may use it to access
homework assignments and check your grades. Various other materials
will be posted there as well either before or after the pertinent
lecture. Lecture notes posted before the lecture may see light
changes after the lecture.


\bigskip
\textbf{\large Prerequisites:}
Stat 401 or an equivalent course is adequate preparation; a grade of B
or better in the previous course is recommended.
Caveat: If it has been a long time since you've had Stat 401 (or equivalent),
you are strongly advised to take Stat 401 before taking
Stat 611.

\bigskip
\textbf{\large Goals and expected learning outcomes:}
Students completing this course will understand the basics of time series analysis,
including, but not limited to, the following:
\begin{enumerate}
\item Carry out exploratory data analysis by plotting data appropriately, using R.
\item Understand, describe, manipulate the key time series models, including auto-regressive
(AR), moving average (MA), ARMA, ARIMA, and SARIMA models.
\item Carry out estimation and forecasting in the time domain using R.
\item Know the basics of maximum likelihood estimation, where applicable to time series.
\end{enumerate}

I intend this course to be roughly equally split between theory and applications.
You will be expected to be able to analyze time series data
using  graphical and analytical
tools that are built into the free software package, R.


\bigskip
\textbf{\large Computing:}
For most of the data analysis in class and homework we will
use the freeware statistics package \texttt{R}.
Your textbook contains most of the \texttt{R} code you will need;
the preface of the book points to additional resources
including the \texttt{R} package \texttt{TSA} and the webpage
http://www.stat.uiowa.edu/$\sim$chan/TSA.htm.
\texttt{R} can be downloaded from http://www.R-project.org; this site
also has a link to an extensive pdf file of documentation.

Depending on the need,
help on the use of \texttt{R} will be provided in lectures
or lecture notes.


\bigskip
\textbf{\large Homework:}

There will be a homework assignment due on most Fridays.
The assignment will posted on the preceding Friday or Saturday.
See ``schedule.pdf'' for details.

Homework should be submitted in class or into the
instructor's mail box no later than 4pm on the due date.

Homework should be typeset/word-processed or hand-written neatly.
Turn in hard copies only. Remember to number the pages and use a stapler.

Late homework will not be accepted in general.
Exceptions are made
on a case-by-case basis by the instructor and typically are made only for
documented health or university-sponsored activities.

\bigskip
\textbf{\large Exams:}

There will be one midterm exam.
Format of the exam will be determined later.
For example, it could be a take-home project.

The two-hour final exam will be comprehensive.
It will be open-book, open-notes.
You may use a calculator, but not a computer, in the final exam.

\bigskip
\textbf{\large Grading policy:}

In all homework and exams,
messy presentation and incomplete/unclear writing, as determined by the
instructor, may cost partial credit.

In all homework and exams,
intermediate steps that show your understanding
of the topic and procedure are as important as the final answer.
Both the procedure and the final answer carry credit.

Your final grade will be calculated based on the following proportions:

\hskip2cm
\begin{tabular}{ll}
Homework & 35\%\\
Midterm & 30\%\\
Final & 35\%
\end{tabular}

Grading scale:
A (honor grade): 90--100;
B (outstanding): 80--89.99;
C (average): 70--79.99;
D (below average): 60--69.99;
F (failure): 0--59.99.

All homework assignments may not be worth the same number of points.

\bigskip
\textbf{\large Ethics}:
Studying together, and getting study assistance from a tutor, is allowed
and encouraged. Copying someone else's work and representing it as your
own is plagiarism. Plagiarism and other forms of cheating in homework
and exams may result in a 0 (zero) score for the homework/exam
involved.

Please read ``Student Code of Conduct'' on pages~117--118 of the
\emph{Class Schedule}, and policies of the Department of Math and Stats
at www.dms.uaf.edu/dms/Policies.html.


\bigskip
\textbf{\large Disability Services}:
If you have a physical handicap or learning disability, please make me
and the Office of Disabilities Services (474-5655) aware of the
situation so that reasonable accommodations can be made.


\bigskip
\textbf{\large Withdrawal:}
I may withdraw any student from class who
(1) misses an exam without a valid reason OR
(2) misses two homework assignments.

\bigskip
\textbf{\large Some Important Dates}:

Please read the inside cover of the ``Class Schedule'' for important
dates regarding adding, dropping, withdrawing, etc.

\bigskip

\textbf{Deadline for drops (no record): Friday, Feb 4.}
\medskip

\textbf{Deadline for withdrawals (`W' grade): Friday, March 25}
\medskip

\textbf{Final exam: 1:00--3:00, Wednesday, May 11, Gruening 309.}

\end{document}
