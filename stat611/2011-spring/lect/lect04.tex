\documentclass[12pt]{article}
\usepackage{coursenote}
\begin{document}
\title{STAT 611 Part 4}
\subtitle{Intervention Analysis}
\maketitle

``Intervention analysis'' is a special (and simple) case of the
so-called ``transfer function models''.
In transfer function models,
the process $Y_t$ is affected by an input process $X_t$,
described by
\[
Y_t = \sum_{j=0}^\infty \tau_j X_{t-j} + N_t
,
\]
where $N_t$ is an ARMA, or ARIMA, or even SARIMA process.
The $\sum_{j=0}^\infty \tau_j X_{t-j}$ part is a deterministic function
(but can use the AR, MA, or ARMA forms of expression).

In intervention analysis, we focus on two types of $X_t$ input:
a \emph{pulse},
\[
P_t^{(T)} =
\begin{cases}
    1,\quad & \text{if $t = T$},\\
    0,      & \text{if $t \ne T$}
\end{cases}
\]
(an instantaneous input at time $T$)
and a \emph{step function}
\[
S_t^{(T)} =
\begin{cases}
    1,\quad & \text{if $t \ge T$},\\
    0,      & \text{if $t < T$}
\end{cases}
\]
(a lasting input starting at time $T$).

Note
\[
P_t^{(T)} = (1 - B) S_t^{(T)}
\]
and
\[
S_t^{(T)}
= (1 - B)^{-1} P_t^{(T)}
= (1 + B + B^2 + B^3 +\dotsb) P_t^{(T)}
\]

\alert
1. We only consider the case where the time of intervention, $T$, is
known.

2. There could be multiple interventions with common or different $T$;
just add them up.

3. The intervention model changes the \emph{mean} of $Y_t$ but not its
variance.

4. $N_t$ is called the ``background'' process (or ``noise'').
It is assumed that the behavior of $N_t$ is unchanged,
hence it is described by one model regardless of the intervention.
The observed process is the background process plus the varying level of
$E(Y_t)$ that is caused by the intervention.


\section{Formulation}

We examine several typical patterns of intervention impact and come up
with expressions for them.
The expressions will contain a small number of parameters,
which will be estimated subsequently.
In other words,
the form of the intervention model is specified based on
\emph{inspection} of graphics of the process
(in contrast to the general ARMA modeling, in which we
``estimate'' the orders before the form of the model can be specified).

Let's write $m_t = \sum_{j=0}^\infty \tau_j X_{t-j}$.

\begin{enumerate}
\item $m_t = \omega P_t^{(T)}$.

    Variation: say there is a $d$ time unit \emph{delay} between the
    intervention event and the appearance of its impact, then
    $m_t = \omega P_{t-d}^{(T)} = \omega B^d P_t^{(T)}$.
\item $m_t = \omega S_t^{(T)}$.

    It's common that the impact of an intervention that happened at time
    $T$ was shown only in $Y_{T+1}$,
    so it's common to see
    $BS_{t}^{(T)}$ or $BP_t^{(T)}$ in the model.

    Figure 11.3(a), p.~253.
\item Figure 11.3(b), p.~253: $m_t$ gradually increases to a stable
    level.

    A geometric series with rate below 1 is the most common way to
    achieve this pattern.
    Think of this general solution:
    \begin{equation}\label{eq:geometric}
    z_n
    = a(1 + b + b^2 + \dotsb + b^n)
    = \frac{a(1 - b^{n+1})}{1 - b}
    \xrightarrow{n \rightarrow \infty} \frac{a}{1 - b},
    \qquad 0 < b < 1
    \end{equation}
    A recursive definition is
    \begin{align*}
        z_0 &= a\\
        z_n &= a + bz_{n-1},\qquad n=1,2,\dotsc
    \end{align*}
    If we want the accumulation to start at time $t$ instead of time 0,
    we can adjust this to
    \begin{align*}
        z_n &= 0, \qquad     &n < t\\
        z_n &= a + bz_{n-1}, &n \ge t
    \end{align*}
    or using the step function,
    \begin{align*}
        z_0 &= a S_0^{(T)}\\
        z_n &= a S_{t}^{(T)} + bz_{n-1},\qquad n=1,2,\dotsc
    \end{align*}

    Or if we want the accumulation to start at time $t+1$,
    and use the textbook notation, we have
    \begin{align*}
        m_0 &= \omega BS_{0}^{(T)}\\
        m_t &= \omega BS_{t}^{(T)} + \delta m_{t-1},
            \qquad t=1,2,\dotsc
    \end{align*}
    that is, $m_0 = 0$ and
    \[
        (1 - \delta B)m_t = \omega B S_{t}^{(T)}
    \]
    or, since we want to express \emph{$m_t$}, we use
    \begin{equation}\label{eq:increase-1}
        m_t = \frac{\omega B}{1 - \delta B} S_t^{(T)}
    \end{equation}

    It might be more intuitive to write this as
    \[
        m_t = (1 + \delta B + \delta^2 B^2 +\dotsb) \omega B S_{t}^{(T)}
    \]
    In this process, we see
    $m_t = 0$,
    $m_{t+1} = \omega$,
    $m_{t+2} = \omega (1 + \delta)$,
    $m_{t+3} = \omega (1 + \delta + \delta^2)$,
    and so on.
    By adjusting $\omega$ and $\delta$,
    we can control the magnitude of the first ``jump''
    and the pace of increase afterwards.

\item Figure 11.3(c), p.~253: $m_t$ increases linearly without bound
(less realistically).

    Just let $b=1$ in~(\ref{eq:geometric}) or $\delta = 1$ in the
    previous definition:
    \[
        m_t = \frac{\omega B}{1 - B} S_t^{(T)}
    \]

\item Figure 11.4(a), p.~254: $m_t$ has a jump after the
intervention, then gradually falls back to the original level.

    This obviously is characterized by
    \begin{align*}
        m_{t} &= 0\\
        m_{t+1} &= \omega\\
        m_{t+k} &= \delta m_{t+k-1},\qquad t>1, 0<\delta<1.
    \end{align*}
    Using the pulse function,
    we can write this as $m_0 = 0$ and
    \[
        m_t = \omega BP_{t}^{(T)} + \delta m_{t-1}
    \]
    or
    \begin{equation}\label{eq:decrease-1}
        m_t = \frac{\omega B}{1 - \delta B} P_t^{(T)}
    \end{equation}
    This is equivalent to
    \begin{align*}
        m_t &= (1 + \delta B + \delta^2 B^2 +\dotsb) \omega B
                P_t^{(T)}\\
            &= \omega \sum_{j=0}^\infty \delta^j P_{t-1-j}^{(T)}\\
            &= \omega \delta^{t-T-1}
                \qquad \text{(b/c we get something only when $t-1-j = T$)}
    \end{align*}
    This process clearly decreases from $\omega$ towards 0 as time
    elapse after $T$.

    On a second thought, we can define
    \[
        m_t = \omega \delta^{t-T} B S_t^{(T)}
    \]
    which seems to be intuitive.

\item Figure 11.4(b), p.~254: jumps and then gradually falls to a level
    above the original.

    Simply add a step function onto the previous one:
    \begin{equation}\label{eq:decrease-2}
        m_t = \frac{\omega_1 B}{1 - \delta B} P_t^{(T)}
                + \omega_2 B S_t^{(T)}
    \end{equation}
    to uniformly lift the post-intervention level.

    This is the same as the one in the book,
    noticing $P_t^{(T)} = (1-B)S_{t}^{(T)}$.

\item Gradually decreases after intervention towards a certain stable level.

    Simply lower~(\ref{eq:decrease-1}) to some level below 0 at time $t+1$:
    \[\begin{split}
        m_t &= \frac{\omega_1 B}{1 - \delta B} P_t^{(T)}
                - (\omega_1 + \omega_2) B S_t^{(T)}
                \\
            &= \frac{\delta \omega_1 - \omega_2
                + (\delta \omega_2 - \omega_1)B}
                {1-\delta B} S_t^{(T)}
                \qquad \text{(using $P = (1-B)S$)}
    \end{split}
    \]

    Better yet, simply negate~(\ref{eq:increase-1}):
    \[
        m_t = -\frac{\omega B}{1 - \delta B} S_t^{(T)}
        ,\qquad \omega > 0, 0<\delta<1.
    \]

\item Figure 11.4(c), p.~254: a one-time jump,
    then dips and gradually rises back.

    Use a pulse to represent the one-time jump,
    followed by a negated~(\ref{eq:decrease-2}):
    \[
        m_t = \omega_0 P_t^{(T)} - \omega_1 B S_t^{(T)}
            - \frac{\omega_2 B}{1 - \delta B} P_t^{(T)}
        ,\qquad \omega_0,\omega_1,\omega_2 > 0; 0<\delta<1.
    \]

\end{enumerate}

\section{Estimation}

After we have specified the model for $m_t$ (with unknown parameters),
the density of $Y_t - m_t$ (that is, $N_t$) can be written.
Hence a MLE can be attempted.

We use explorations of the series prior to the intervention
for tentative specification of model for $N_t$.

If the process looks stationary after removing the intervention effect,
we could try a simple least squares estimation of the intervention model
parameters.
After that, get the residual series and check what model is appropriate
for it.
(The book does not mention this strategy, and I didn't try it.)

As usual,
after estimation of the full model,
various diagnostics should be checked,
and the model revised (and refitted, and re-checked) if necessary.

\end{document}

