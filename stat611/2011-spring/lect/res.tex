\documentclass[12pt]{article}
\usepackage{coursenote}
\begin{document}
\title{STAT 611 Chapter 3}
\maketitle


\section{Residual analysis}

Looking at residual plots is a basic and very effective way of spotting
problems, meaning, possible violation of model assumptions in the data.

Residuals:
\[
\hat{X}_t = Y_t - \hat{\mu}_t
\]
``Standardized'' or ``studentized'' residuals provide more accurate
info.

Several things to check:
\begin{enumerate}
\item $t$ versus $\hat{X}_t$

    Does it look random? Is there any systematic behavior or pattern,
    such as continued positive or negative residuals, or cyclical
    behavior? Outliers?
    Patterns will defy independence.
\item $Y_t$ versus $\hat{X}_t$

    Again, any systematic pattern?
\item Normality plot: QQ plot.

    Quantiles of $\hat{X}_t$ versus
    corresponding theoretical quantiles of a normal distribution.
\item Sample autocorrelation function and correlogram

    (3.6.2), p.~46

    Behavior of the correlogram: pace of decay, cyclic?
\end{enumerate}

\end{document}
