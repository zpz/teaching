\documentclass{article}
\usepackage{techart}
\parindent=0pt
\parskip=5pt
\usepackage{ulem}

\begin{document}
\title{STAT F621: Distribution-Free Statistics, Fall 2011}
\author{MWF 2:15--3:15 Gruening 308}
\date{}
\maketitle

\textbf{\large Instructor: Zepu Zhang.}

Office location: 306A Chapman.

Contact: zzhang6@alaska.edu, 474-7605.  Mail box in Chapman 101.

Office Hours: MWF 4:00--5:00; or by appointment.

\bigskip
\textbf{\large Textbook}:
\textit{Practical Nonparametric Statistics},
by W.\@ J.\@ Conover, 3rd ed, 1999.

\bigskip
\textbf{\large Blackboard}:
A course Blackboard site has been set up. You may use it to access
homework assignments, solutions, and check your grades.
Various other materials will be posted there as well either before or after the pertinent
lecture.

Be sure to check the "Announcement" once you have logged into the course
website. Newly posted or updated materials are usually announced there.

\bigskip
\textbf{\large Goals and expected learning outcomes:}
Methods for distribution-free (nonparametric) statistical estimation and
testing. These methods apply to many practical situations including
small samples and non-Gaussian error structures.
This course focuses on the more ``classical'' of these methods,
making heavy use of binomial distribution, $\chi^2$ distribution, ranks, etc.
The methods will be presented and illustrated using a variety
of applications and data sets.

\bigskip
\textbf{\large Prerequisites:}
STAT F200X [STAT S273-J].

\bigskip
\textbf{\large Computing:}
The amount of computation in the homework justifies using a computer package.
You will deal with a lot of data manipulation especially ordering
numbers, finding out the rank of each element in a data set, and such.
Other than that you will need to do some routine math operations such as
squaring, adding things up, etc.
You may also want to use functions related to standard distributions
such as the normal distribution and $\chi^2$ distribution.
Such functions are more convenient than traditional table look-up.
(But you still need to know how to use the tables.)

We will use \texttt{R} in this course.
Help on \texttt{R} in lectures will be offered on an as-needed basis.
Some materials are posted on Blackboard to help you get started with
\texttt{R}.

\bigskip
\textbf{\large Homework:}
\emph{Reading relevant sections of the textbook} both before and after
the lecture will be very helpful to your understanding.
Take the reading as assignment and finish it in a timely manner.

Homework due dates are listed in the tentative ``schedule''.
Homework should be \emph{submitted in class or into the
instructor's mail box in Chapman 101.
To avoid being late due to Chapman building closure,
play it safe and turn it in by 4pm.}

\emph{Late homework will not be accepted in general.}
Exceptions are made
on a case-by-case basis by the instructor and typically are made only for
documented health or university-sponsored activities.

\textbf{Homework guidelines}

(Some of the following guidelines also apply to exams.)

\begin{enumerate}
\item Presentation counts.
    \begin{enumerate}
    \item You should \emph{work out the problems first on scratch paper},
        then copy down your answers in a clean, organized way.
    \item \emph{Prepare your final answer on a computer and print it
        out.}
    \item If you must hand-write your homework, make sure you copy down
        worked answers in a clean, neat, orderly form.
        Avoid the freedom of having chunks of things floating around.
        Do it as if you are using a computer so that the content has to
        flow from top to bottom.
        \emph{Highlight key answers in boxes}.
    \item \emph{Do not shuffle the problems.}
    \item \emph{Messy presentation}, as determined by the grader, may cost you
        partial credit.

        Incomplete or unclear writing (referring to the flow of ideas
        rather than the strokes),
        as determined by the grader, may cost you partial credit.
    \end{enumerate}
\item \emph{Show your work!} Include the \emph{right amount of detail}:
    \begin{enumerate}
    \item When you use a formula, always show the general formula (with
        math symbols) before plugging in actual numbers.
    \item After you have plugged in numbers, be concise with routine arithmetic.
    \item Include important middle steps, formulas, intermediate
        quantities.
    \item Lack of important middle steps will cost you partial or whole
        credit.
    \end{enumerate}
\item Turn in hard copies only. Remember to number the pages and use a stapler.
\end{enumerate}

\bigskip
\textbf{\large Exams:}
There will be a take-home midterm exam
and a final project and presentation as final exam.

\bigskip
\textbf{\large Overall grade:}
Your final grade will be calculated based on the following proportions:

\hskip2cm
\begin{tabular}{ll}
Homework & 50\%\\
Midterm & 20\%\\
Final & 30\%
\end{tabular}

Grading scale:

\hskip2cm
\begin{tabular}{ll}
A (outstanding) & 90--100\\
B (good)  & 80--89.99\\
C (satisfactory) &  60--79.99\\
F (failure) & 0--59.99
\end{tabular}

All homework assignments may not be worth the same points due to their
difference in work-load and difficulty.

\bigskip
\textbf{\large Ethics}:
Studying together, and getting study assistance from a tutor, is allowed
and encouraged. Copying someone else's work and representing it as your
own is plagiarism. Plagiarism and other forms of cheating in homework
and exams may result in a 0 (zero) score for the homework/exam
involved.

Please read ``Student Code of Conduct''
and policies of the Department of Math and Stats at\\
www.dms.uaf.edu/dms/Policies.html.
(The ``Code of Conduct'' used to be in the ``Class Schedule''.
Now that the schedule has gone digital, I could not find a link to the
complete ``Code''.)


\bigskip
\textbf{\large Disability Services}:
If you have a physical handicap or learning disability, please make me
and the Office of Disabilities Services (474-5655) aware of the
situation so that reasonable accommodations can be made.


\bigskip
\textbf{\large Withdrawal:}
I may withdraw any student from class who
(1) misses an exam without a valid reason OR
(2) misses two homework assignments.

\end{document}
