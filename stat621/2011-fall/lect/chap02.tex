\documentclass[12pt]{article}
\usepackage{coursenote}
\begin{document}
\title{STAT 621 Chapter 2}
\maketitle


\section{Measurement scales}

\begin{itemize}
\item Nominal scale
\item Ordinal scale
\item Interval scale
\item Ratio scale
\end{itemize}

\section{Random samples}

\emph{Def 1}: p.~70, via ``equally likely draws''

\emph{Def 2}: p.~71, via i.i.d random variables

\emph{with replacement} vs \emph{without replacement}

Effectively ``without replacement'':
a small sample without replacement out of a large population.

\emph{statistic}: a function of a random sample

a random sample viewed as a multivariate random variable,
which has a corresponding sample space

\emph{statistic} viewed as a random variable:
it assigns a value to each point in the sample space
of the random sample (which is a multivariate random variable)

\emph{order statistic}

\section{Estimation}

\subsection{Empirical distribution function}

\example Ex.~1, p.~79.

sample quantiles

\subsection{Estimator, estimate}

sample mean

sample variance, sample standard deviation

distribution of an estimator (which is a statistic, i.e.\@ random
variable)

mean of an estimator, \emph{unbiased estimator}

variance of an estimator, \emph{standard error}

point estimator vs interval estimator

\subsection{Confidence interval}

approx CI based on CLT, large $n$ (say 30)

\example Ex.~5, p.~86.

\subsection{Bootstrap}

``computer-intensive statistics''

\subsection{Summary}

parameter estimation in general (p.~88)

\section{Hypothesis testing}

\subsection{Basic terminology (p.~95--96)}

alternative hypothesis\\
null hypothesis\\
test statistic\\
decision rule\\
accept, or fail to reject, the null

\example Ex.~1, p.~96

simple/composite hypothesis (p.~97)

critical region, rejection region (p.~98)\\
two-tailed tests,
one-tailed tests,
upper-tailed tests, lower-tailed tests

\subsection{Error types}

\emph{Type I error}: def~4, p.~98.

\emph{Type II error}: def~5, p.~99.

\emph{Level of significance}, $\alpha$: def~6, p.~99.

\emph{Power}, $1-\beta$: def~8, p.~100.

Comparison of the two types of errors:
known or unknown? unique? controlled by the user?
(These questions have to do with whether the hypotheses are simple or
composite.)


\subsection{Null distribution}

def~7, p.~989.

\subsection{$p$-Value}

def~9, p.~101.

\example Ex.~2, p.~101.


\subsection{Some properties of hypothesis tests}

\emph{Power function}

\emph{Relative efficiency}

\section{Remarks on parametric vs nonparametric methods}

p.~114--118.

A definition of nonparametric methods: p.~118.
\end{document}
