\documentclass[12pt]{article}
\usepackage{coursenote}
\begin{document}
\title{STAT 621 Chapter 3}
\maketitle

\section{The Binomial test}

Highlights:
\begin{enumerate}
\item Recognize the generality and broad applicability of Binomial-based
tests.
\item The test statistic ($T$) and its null distribution.
\item Normal approximation to binomial.
    \begin{itemize}
    \item ``large sample size''
    \item ``correction for continuity''
    \end{itemize}
\item Alternative hypotheses: two-tailed vs one-tailed.
\item Implications of the discrete nature of $T$:
    \begin{itemize}
    \item determination of the rejection region
    \item the resultant level of significance ($\alpha$)
    \item to include or not to include ``or equal to''
    \end{itemize}
\item How to determine the $p$-value.
\end{enumerate}

\example Ex.~1, page 127. (Point to learn: how to state (choose) $H_0$
and $H_1$.)

\example Ex.~2, page 128.

\section{Confidence interval for a probability or population proportion
($p$)}

Highlights:

\begin{enumerate}
\item Method A (exact) vs Method B (approx).
\item Derivation of the formulas for Method B.
\item How was the table for Method A made?
\end{enumerate}

\example Ex.~3, page 130.

\section{The sign test}

The sign test is the oldest of all nonparametric tests,
and it is a special Binomial test. While a Binomial test tests whether
$p$ is a certain value, say $p_*$,
the sign test tests whether $p$ is 0.5.

Specifically,
there exist a sample of a bivariate variable:
$(x_1, y_1)$, $(x_2, y_2)$,..., $(x_n, y_n)$.
We want to test whether $X$ on average tends to be larger than $Y$.

Highlights:
\begin{enumerate}
\item Convert a bivariate problem to a univariate problem---let
    $Z = X - Y$, then the bivariate sample induces a univariate sample
    $z_1,\dotsc, z_n$.
    The test becomes whether $Z$ tends to be positive or negative.
\item Get the sign ($+$ or $-$) of $z_i$.
    Then the test becomes whether there tend to be more $+$'s than
    $-$'s. Or, in other words, whether $P(+) = 0.5$.
    Hence this is a Binomial test with $T = \#\{+\}$ and $p_* = 0.5$.
\item Discard ties; in other words, $z=0$ contains no info about the
    tendency of the sign of $Z$.
\item Independent sample: independence between pairs.
\item ``Internally consistent'': if $P(+) > P(-)$ for one pair, then
    this is the case for all pairs.
    This is one of the requirement of a Binomial experiment: constant
    $p$ in all trials.
\item How to state the hypothesis: $P(+) = P(-)$ vs alternatives.
\item An alternative statement of hypothesis: $E(X) = E(Y)$ vs
    alternatives.

    This involves some assumption about the distribution of $Z = X - Y$.
    Specifically, it assumes the distribution of $Z$ is symmetric about
    its mean.
\end{enumerate}


\section{Variation 1 of the sign test: McNemar test for significance of
change}

The value of some attribute is one of two categories; let's call them
``0'' and ``1''.
We take a sample of size $n'$; some have value ``0'' and the others have
value ``1''.
After a certain incident,
the values of this sample is measured again: some will stay the same as
before, some ``0'' switched to ``1'', and some ``1'' switched to ``0''.
The counts of ``0''s and ``1''s can be summarized in a \emph{contingency table}:

\begin{center}
\begin{tabular}{cc|cc}
    &   & after & \\
    & & 0   & 1 \\ \hline
before & 0 & a & b \\
    & 1 & c &d \\ \hline
\end{tabular}
\end{center}

The question is: have the fractions of ``0''s and ``1''s changed?

The test is really whether $b = c$.

Exact test when $b + c \le 20$: use Binomial (or sign) test.

Large sample test when $b + c > 20$: use normal approx.

\example Ex.~1, page 168.

\section{Variation 2 of the sign test: Cox \& Stuart test for trend}

\example Ex.~2, page 171 (the simple case)

\example Ex.~3, page 171 (filtering out periodicity)

\example Ex.~4, page 172 (detecting correlation)

\example Ex.~5, page 173 (testing for the presence of a pattern of
relation)

\end{document}
