\documentclass[12pt]{article}
\usepackage{techart}
\begin{document}
\title{STAT 621 HW 1}
\author{Due Monday, 10/3}
\maketitle

In your homework writeup,
write formulas, explanations, etc., just like any other homework you do,
but use \texttt{R} as much as possible in actually
carrying out your computations,
for example sorting, listing combinations, factorial, ``choose''.
Use this opportunity to discover \texttt{R} functions for your
particular computational need.
Include your \texttt{R} code and an appropriate amount of output
in the writeup.

\begin{enumerate}
\item
    To play the lottery game Texas Lotto,
    a player selects six of the fifty numbers from 1 to 50,
    without replacement (i.e., without selecting the same number twice).
    The Lottery Commission then selects six of the numbers from 1 to 50
    at random (i.e., every six-number combination is equally likely to
    be selected), also without replacement.
    The player wins if at least three of the player's numbers match
    the numbers drawn by the Lottery Commission, in any order.

    Let $X$ equal the number of ``matched numbers'' achieved by the
    player. The player wins if $X$ equals 3, 4, 5, or 6.

    \begin{enumerate}
    \item
        How many combinations of six numbers out of 50 are there,
        disregarding the order in which they were drawn?
        What is the possibility of each combination when the numbers are
        drawn at random?
    \item
        What is the probability that all six of the player's numbers
        match the six drawn at random by the Lottery Commission?
        That is, what is $P(X=6)$?
    \item
        Find the probabilities $P(X=5)$, $P(X=4)$, and $P(X=3)$.
    \item
        One drawing produced 241,024 tickets that had exactly three
        numbers correct (i.e., $X=3$).
        How many tickets do you think were purchased for that drawing?
    \item
        The same drawing referred to in the previous question produced
        12,422 tickets that had exactly four numbers correct.
        Is this consistent with the number of tickets that had exactly
        three numbers correct?
    \item
        Use computer simulation to empirically verify
        the probabilities that $X$ equals 3, 4, 5, and 6.
    \end{enumerate}

\item
    Seven male students have interviewed for three positions
    as summer camp counselors. The students are ranked according to
    height, from 1 (tallest) to 7 (shortest).
    The null hypothesis is that each student is equally likely to be
    selected, while the alternative hypothesis is that the three taller
    students are twice as likely to be selected as the four shorter
    students. Assume the students are selected independently of each
    other. The test statistic is the sum of the ranks of the three
    students who were selected. The decision rule is to reject the null
    hypothesis if the test statistic is 6 or less.
    \begin{enumerate}
    \item
        Is the null hypothesis simple or composite?
    \item
        Find the level of significance.
    \item
        Find the power.
    \item
        Find the null distribution of the test statistic,
        that is, list the probability of each possible value of the test
        statistic.
        (Find a way to let \texttt{R} list this for you.)
    \item
        Find the ``alternative distribution'' of the test statistic.
    \end{enumerate}
\end{enumerate}

\end{document}
