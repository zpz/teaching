\documentclass{article}
\usepackage{techart}
\parindent=0pt
\parskip=5pt

\begin{document}
\title{STAT 651: Statistical Theory I}
\author{Lectures: MWF 10:30--11:30 Gruening 309\\
Final exam: 10:15--12:15, Fri Dec.~17, Gruening 309}
\date{}
\maketitle

\textbf{\large Instructor: Zepu Zhang.}

zzhang6@alaska.edu;
306A Chapman, 474-7605.
Mail box: Chapman 101.

Office Hours: MWF 2:30--3:30,
or by appointment.

\bigskip
\textbf{\large Textbook}:

Required:\\
\textit{Statistical Inference}, 2nd edition,
by George Casella and Roger Berger,
2002, Duxbury.

Reference:\\
\textit{Probability and Statistical Inference},
by Nitis Mukhopadhyay, 2000, Marcel Dekker.

\bigskip
\textbf{\large Blackboard}:
Use the Blackboard site for this course to access
schedule, announcements, lecture notes, homework assignments and
solutions, grades, and other related materials.
Some documents may see small modifications/updates after first posting.

\bigskip
\textbf{\large Prerequisites:}
Math 202X, Math 314, previous statistics course, or permission of
instructor. In particular, this course will put to use a great deal of
the calculus you have learned in your earlier coursework.

\bigskip
\textbf{\large Goals and expected learning outcomes:}
Students will gain an understanding of basic probability theory, random
variables and their properties, random samples, and convergence
concepts. Throughout, emphasis will be placed on reading and writing
rigorous mathematical proofs, as well as understanding technical writing
in statistics. Students will be introduced to simulations by the
programming language \texttt{R}.

\bigskip
\textbf{\large Computation and software:}
This is a theory class and so computation will not play a large role.
However we will use simulation to illustrate certain concepts.
Students will use \texttt{R} which is available in the computer lab in
room 407 of the Bunnell building and in the math lab in room 305 of the
Chapman building.
You may also download \texttt{R} for free at \texttt{www.r-project.org}.
Learning to use \texttt{R} is not part of this course;
but the instructor can provide help if needed.

\bigskip
\textbf{\large Lecture notes:}
Lecture notes serve as outlines and pointers.
They are not meant to be a polished reader.
Unless otherwise announced,
the notes mention all required topics,
and textbook topics not mentioned in the notes are less important.
It may happen that certain content of the notes
do not get enough discussion time in the lectures;
that content is still required material.

\bigskip
\textbf{\large Homework:}
I will use Blackboard (\texttt{classes.uaf.edu}) to assign readings from
the text and exercises.
Assignments should be submitted by 5pm on the date due.
I generally do not accept late homework.
If you will not be able to complete an assignment on time,
see me \emph{before} the due date to make arrangements.
Permitted late submission will lose partial credit.
I encourage you to discuss homework problems with other students,
as well as with me.
However the work you turn in must be your own.

Homework should be typeset/word-processed (preferred) or hand-written neatly.
Turn in hard copies only. Remember to number the pages and use a stapler.

When writing up your homework, please use complete sentences.
State each problem (briefly, if not in full detail) before solving it;
this is a good habit to get into,
and it makes it possible for you to read and make use of your homework
later on. When proving facts, be sure to state the assumptions, present
your argument in a logical order, and point out where you make use of
the various assumptions.
Properly cite any theorems or definitions used.

Include relevant computer output with your solutions.
Decorate your \texttt{R} output---with circles and arrows and a few
words point out where your answer is.


\bigskip
\textbf{\large Grading policy:}

There will be two in-class hour-long midterm exams
and one two-hour final exam.
Coverage of each exam will be announced later;
the final exam will in principal cover the entire course.

The exams will be closed-book. You may use notes \emph{prepared by
yourself} on two sheets of $8 \frac{1}{2} \times 11$ inch paper (both sides).
You may not use a computer in the exams.
You may use a calculator.
(It must be a calculator only; it must not be a device that can store
notes.)

In all homework and exams,
messy presentation and incomplete/unclear writing, as determined by the
grader, may cost partial credit.

In all homework and exams,
intermediate steps that show your understanding
of the topic and procedure are as important as the final answer.
Both the procedure and the final answer carry credit.

Your final grade will be calculated based on the following proportions:

\hskip2cm
\begin{tabular}{ll}
Homework & 30\%\\
Midterms & 40\% ($20\% \times 2$)\\
Final & 30\%
\end{tabular}

Grading scale:
A (honor grade): 90--100;
B (outstanding): 80--89.99;
C (average): 70--79.99;
D (below average): 60--69.99;
F (failure): 0--59.99.

\bigskip
\textbf{\large Ethics}:
Studying together, and getting study assistance from a tutor, is allowed
and encouraged. Copying someone else's work and representing it as your
own is plagiarism. Plagiarism and other forms of cheating in homework
and exams may result in a 0 (zero) score for the homework/exam
involved.

Please read ``Student Code of Conduct'' on pages~117--118 of the
\emph{Class Schedule}, and policies of the Department of Math and Stats
at www.dms.uaf.edu/dms/Policies.html.


\bigskip
\textbf{\large Disability Services}:
If you have a physical handicap or learning disability, please make me
and the Office of Disabilities Services (474-7043) aware of the
situation so that reasonable accommodations can be made.


\bigskip
\textbf{\large Withdrawal:}
I may withdraw any student from class who
(1) misses an exam without a valid reason OR
(2) misses three homework assignments.

\end{document}
