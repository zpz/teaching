\documentclass[12pt]{article}
\usepackage{techart}
\begin{document}
\title{STAT 651 HW 3 Solution}
\maketitle

\begin{enumerate}
\item (5 + 10 points)
\begin{enumerate}
\item[(a)]
We show the independence of $A$ and $\emptyset$ by the definition of
independence, i.e.\@ $P(A\cap \emptyset) = P(A) P(\emptyset)$.

\[
P(A \cap \emptyset)
= P(\emptyset) P(A \given \emptyset)
= 0\cdot P(A \given \emptyset)
= 0
= P(A)P(\emptyset)
\]

\item[(b)]
\[\begin{split}
P(A^c \cap B^c)
&= P\bigl((A \cup B)^c\bigr)
\\
&= 1 - P(A \cup B)
\\
&= 1 - P(A) - P(B) + P(A \cap B)
\\
&= 1 - P(A) - P(B) + P(A)P(B)
\\
&= \bigl(1 - P(A)\bigr) \bigl(1 - P(B)\bigr)
\\
&= P(A^c) P(B^c)
\end{split}
\]
hence $A^c$ and $B^c$ are independent.
In the proof we used
$P(A\cap B) = P(A)P(B)$, because $A$ and $B$ are independent.
\end{enumerate}

\item (10 + 10 + 10 points)

\begin{enumerate}
\item[(a)]
By Theorem~1.5.3,
we show $F(x)$ is a cdf by verifying the following facts:

\begin{enumerate}
\item[(1)] $\lim_{x\to -\infty} F(x) = 0$,
$\lim_{x\to \infty} F(x) = 1 - \lim_{x\to \infty} \exp(-ax) = 1$.

\item[(2)] $F(x)$ is non-decreasing, because

(a) On $(-\infty, 0)$, $F(x) = 0$.

(b) On $[0, \infty)$,
$\frac{\diff F(x)}{\diff x} = a e^{-ax} > 0$,
given $a > 0$.

(c) At $x=0$, $F(x) = 1 - \exp(0) = 0 \ge F(0-)$.
\item[(3)]
$F(x)$ is right-continuous,
because $F(x)$ is continuous on
$(-\infty, 0)$ and $[0, \infty)$ respectively,
and is continuous at $x=0$ because
$F(0) = 0$ while $F(0-) = 0$.
\end{enumerate}

\item[(b)]
The pdf is
\[
f(x) = \begin{cases}
    0,          & x < 0\\
    a e^{-ax},  & x \ge 0
\end{cases}
\]

\item[(c)]
On $(-\infty, 0)$,
$F_Y(y) = 0$.

On $[0, \infty)$, by Theorem~2.1.3,
$
F_Y(y)
= F_X(\sqrt{y})
= 1 - e^{-a\sqrt{y}}
$.
\end{enumerate}

\item (40 points)

Similar to problem 2(a). Show the three properties listed in
Theorem~2.1.3.

To show non-decreasing, we do not have to show the derivative (although
that is the standard way). We could simply say, for example,
$1/2 + 1/\pi \tan^{-1}(x)$ is an increasing function because
$\tan^{-1}(x)$ is,
or $\bigl(1 + \exp(-x)\bigr)^{-1}$ is an increasing function
because $1 + \exp(-x)$ is positive and decreasing.

To show right-continuity of a function that is defined by a single
formula on the whole $(-\infty, \infty)$,
we could simply point out that the function is a continuous one.

Details are omitted.

\item (3 + 3 points)

\begin{enumerate}
\item[(a)]
\[
c
= \biggl(\int_{0}^{\pi/2} \sin x\diff x\biggr)^{-1}
= \biggl(-\int_{0}^{\pi/2} \diff \cos x\biggr)^{-1}
= \bigl(-(0 - 1)\bigr)^{-1}
= 1
\]
\item[(b)]
\[
c
= \biggl(\int_{-\infty}^{\infty} \exp(-|x|) \diff x\biggr)^{-1}
= \biggl(2\int_{0}^{\infty} \exp(-x)\diff x\biggr)^{-1}
= \biggl(2\exp(-x)\Bigm|_{\infty}^0\biggr)^{-1}
= 1/2
\]

\end{enumerate}

\item (6 + 6 points)

\begin{enumerate}
\item[(a)]
\[
f_Y(y)
= f_X(y^{1/3}) \cdot \frac{1}{3} y^{-2/3}
= 42\bigl(y^{1/3}\bigr)^5 \bigl(1 - y^{1/3}\bigr)
    \cdot \frac{1}{3} y^{-2/3}
= 14y\bigl(1 - y^{1/3}\bigr)
,\quad
0 < y < 1
\]

\[
\int_0^1 f_X(x)\diff x
= 42\int_0^1 x^5(1-x) \diff x
= 42\biggl(\int_0^1 x^5\diff x - \int_0^1 x^6\diff x\biggr)
= 42\Bigl(\frac{1}{6} - \frac{1}{7}\Bigr)
= 1
\]

\[
\int_0^1 f_Y(y) \diff y
= 14 \int_0^1 y\bigl(1 - y^{1/3}\bigr) \diff y
= 14\biggl(\int_0^1 y\diff y - \int_0^1 y^{4/3}\diff y\biggr)
= 14\Bigl(\frac{1}{2} - \frac{3}{7}\Bigr)
= 1
\]


\item[(b)]
\[
f_Y(y)
= f_X\Bigl(\frac{y - 3}{4}\Bigr) \cdot \frac{1}{4}
= 7\exp\Bigl(-7\frac{y-3}{4}\Bigr) \cdot \frac{1}{4}
= \frac{7}{4} \exp\Bigl(-\frac{7}{4}(y-3)\Bigr)
,
\quad
3 < y < \infty
\]
($X$ is exponential; $Y$ is shifted exponential.)

\[
\int_0^\infty f_X(x)\diff x
= \int_0^\infty 7 e^{-7x}\diff x
= -e^{-7x}\Bigm|_0^\infty
= -(0 - 1)
= 1
\]

\[
\int_3^\infty f_Y(y)\diff y
= \int_3^\infty \frac{7}{4} e^{-\frac{7}{4}(y-3)} \diff y
= \int_0^\infty \frac{7}{4} e^{-\frac{7}{4}x} \diff x
= -e^{-7x/4}\Bigm|_0^\infty
= -(0 - 1)
= 1
\]

\end{enumerate}

\end{enumerate}



\end{document}
