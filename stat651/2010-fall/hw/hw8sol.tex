\documentclass[12pt]{article}
\usepackage{techart}
\parindent=0pt
\parskip=8pt
\begin{document}
\title{STAT 651 HW 8 Solution}
\maketitle

Note in the solution of question~1 a common strategy in math derivations:

\underline{Delay number plug-in as much as possible}.

It has two advantages:
(1) The result is more general---it is general until you plug in
numbers. If you need to plug in different sets of numbers,
there are fewer steps of calculation you need to do.
(2) It's easier to check if something goes wrong.

Try to use this strategy in all your math work.
Not only with number plug-in.
Work in a general form as far as you can.

\begin{enumerate}
\item
\begin{enumerate}
\item
\[
f_X(x)
= f(x, 1) + f(x, 2)
= \frac{2x+3}{21}
= \begin{cases}
        5/21 & x = 1\\
        1/3 & x = 2\\
        3/7 & x = 3\\
        0 & \text{otherwise}
    \end{cases}
\]
\[
f_Y(y)
= f(1, y) + f(2, y) + f(3, y)
= \frac{y+2}{7}
= \begin{cases}
        3/7 & y=1\\
        4/7 & y=2\\
        0 & \text{otherwise}
    \end{cases}
\]
\item
$P(X \le 2) = f_X(1) + f_X(2) = 4/7$.
$P(Y > 1) = f_Y(2) = 4/7$.
\item
\[
f(x\given y)
= \frac{f(x,y)}{f_Y(y)}
= \frac{f(x,y)}{f(1,y) + f(2,y) + f(3,y)}
= \frac{x+y}{(1+y) + (2+y) + (3+y)}
= \frac{x+y}{6 + 3y}
\]
$x=1,2,3$; $y=1,2$.
After this point, you may proceed to get specific numbers.
\[
f(y\given x)
= \frac{f(x,y)}{f_X(x)}
= \frac{f(x,y)}{f(x,1) + f(x,2)}
= \frac{x+y}{2x + 3}
\]
$x=1,2,3$; $y=1,2$.
\end{enumerate}

\item
\begin{enumerate}
\item
\[
f_X(x) = \int_0^x 4e^{-2x} \diff y = 4xe^{-2x},
\quad 0<x<\infty
\]
\[
f(y\given x)
= \frac{f(x,y)}{f_X(x)}
= \frac{4e^{-2x}}{4xe^{-2x}}
= \frac{1}{x},
\quad
0 < y < x < \infty
\]
This shows that $Y$ is uniform on $(0,x)$ for any specific $x$.

$E(Y\given x) = x/2$.\\
$E(Y^2\given x) = \int_0^x y^2\cdot (1/x) \diff y = x^2/3$.\\
$\var(Y\given x) = E(Y^2\given x) - \bigl(E(Y\given x)\bigr)^2 =x^2/12$.

\item
For uniform on $(a,b)$, we know the mean is $(a+b)/2$ and the variance
is $(b-a)^2/12$. Now $a=0$ and $b=x$.

\item
\[
f_Y(y)
= \int_y^{\infty} 4e^{-2x} \diff x
= 2e^{-2y}
,\quad
0<y<\infty
\]
This is exponential.
$E(Y) = 1/2$.
$\var(Y) = 1/4$.
We see
$\var(Y) < \var(Y\given x)$ when
$x < \sqrt{3}$ and
$\var(Y) \ge \var(Y\given x)$ otherwise.
This is understandable by looking at the support of the joint distribution.
\end{enumerate}

\item
\begin{enumerate}
\item
\[
f_X(x) = \int_0^1 2(1-x) \diff y = 2(1-x),\quad 0<x<1
\]
\[
f_Y(y) = \int_0^1 2(1-x) \diff x = 1,\quad 0<y<1
\]
\item
$E(X) = \int_0^1 2x(1-x) \diff x = 1/3$.
$E(X^2) = \int_0^1 2x^2(1-x) \diff x = 1/6$.
$\var(X) = 1/18$.

\item
$Y$ is uniform on $(0,1)$.
$E(Y) = .5$. $\var(Y) = 1/12$.

\item
\[
f(x\given y)
= \frac{f(x,y)}{f_Y(y)}
= 2(1-x)
,\quad
0<x<1
\]
\[
f(y\given x)
= \frac{f(x,y)}{f_X(x)}
= \frac{2(1-x)}{2(1-x)}
= 1
\]
(Notice the joint density.
Imagine you cut at a certain $x$ to get the profile of
$f(y\given x)$---the profile does not depend on where you cut.
Similarly for $f(x\given y)$.)

\item
$E(X\given y) = \int_0^1 2x(1-x)\diff x = 1/3$.\\
$E(X^2\given y) = \int_0^1 2x^2(1-x) \diff x = 1/6$.\\
$\var(X\given y) = 1/18$.
\end{enumerate}

\item
\begin{enumerate}
\item
\[
f_X(x) = \int_0^x 3x\diff y = 3x^2,
\quad 0<x<1.
\]
\[
f_Y(y) = \int_y^1 3x \diff x = \frac{3}{2}(1 - y^2),
\quad 0<y<1.
\]

\item
\[
f(x\given y)
= \frac{f(x,y)}{f_Y(y)}
= \frac{3x}{(3/2)(1-y^2)} = \frac{2x}{1-y^2}
,\quad 0<y<x<1
\]
\[
f(y\given x)
= \frac{f(x,y)}{f_X(x)}
= \frac{3x}{3x^2}
= \frac{1}{x}
,\quad 0<y<x<1
\]

\item
Noticing $Y\given X=x \sim \text{uniform}(0,x)$,
$E(Y \given x) = x/2$.
$\var(Y\given x) = x^2/12$.

\end{enumerate}

\item
From a statistical perspective,
these questions involve integrating the joint density
over the region that satisfies the condition.
However, since the density is flat,
the probability is simple the proportion of the region of interest
in the entire support.
\begin{enumerate}
\item  $\pi/4$.
\item The condition is $Y < 2X$. This region is half of the support.
Note as long as you cut through the center,
the square is divided into two equal parts.
The probability is $1/2$.
\item The entire support satisfies this condition, therefore the
probability is 1.
\end{enumerate}

\item
\begin{enumerate}
\item Integral of the density should be 1. From that we find $C = 1/4$.
\item
\[
f_X(x)
= \int_0^1 \frac{x+2y}{4} \diff y
= \frac{x+1}{4},
\quad 0<x<2
\]
\item
\[
F(x,y)
= \frac{1}{4} \int_{-\infty}^x \int_{-\infty}^y (u + 2v) \diff v \diff u
= \begin{cases}
    0, & x \le 0 \text{ or } y \le 0\\
    \frac{1}{8}(x^2 y + 2xy^2), & 0<x\le 2, 0<y\le 1\\
    \frac{1}{8}(x^2 + 2x), & 0<x\le 2, 1<y\\
    \frac{y^2 + y}{2}, & 2<x, 0<y\le 1\\
    1, & 2<x, 1<y
    \end{cases}
\]
\end{enumerate}

\item
\begin{enumerate}
\item Rows and columns with 0's are problematic.
The easiest argument is like this:
$f(1, 4) = 0$ but obviously $f_X(1) > 0$ and $f_Y(4) > 0$,
hence $f(1,4) \ne f_X(1) f_Y(4)$.

\item Find the marginals of $X$ and $Y$, which are also marginals of $U$
and $V$.
\[
f_X(x) = 1/4, 1/2, 1/4, \quad \text{for $x=1,2,3$, respectively}
\]
\[
f_Y(y) = 1/3, \quad \text{for $y=2,3,4$}
\]
Then find the joint $f(u,v) = f_U(u) f_V(v)$ for each combination of $u$
and $v$ and make a table.
\end{enumerate}
\end{enumerate}

\end{document}
