\documentclass[12pt]{article}
\usepackage{techart}
\parskip=8pt
\parindent=0pt

\begin{document}
\title{STAT 651 HW 7 Solution}
\maketitle

\begin{enumerate}
\item Describe the scale-location family associated with the standard pdf
given by $f(x) = \frac{1}{2}\exp(-|x|)$, $-\infty < x < \infty.$

\texttt{The family of distributions are the densities
$f(x) = \frac{1}{2\sigma} \exp\Bigl(-\frac{|x - \mu|}{\sigma}\Bigr)$,
$\mu \in R$, $\sigma > 0$, $x \in R$.
}


\item Let $X$ be a random variable having
pdf $f_X(x) = c \exp(-x/3) x^d I(0\le x < \infty)$,
where $c$ and $d$ are positive constants.
If $E(X) = 7$ and $\var(X) = 21$, then:
  \begin{enumerate}
  \item Find $c$ and $d$.  Identify the distribution by name.

  \texttt{$X$ has a gamma distribution with parameters
  $\alpha = d+1$ and $\beta = 3$.
  From $E(X) = \alpha \beta = 3\alpha = 7$ we get $\alpha = 7/3$,
  and this is consistent with the other condition since
  $\var(X) = 7 = \alpha \beta^2 = \frac{7}{3} 3^2 = 21$.
  \[c = \frac{1}{\Gamma(\alpha)\beta^{\alpha}}
    = \frac{1}{\frac{4}{3}\frac{1}{3}\Gamma(1/3) 3^{7/3}}
    = \frac{1}{4\cdot 3^{1/3}\Gamma(1/3)}
  \,\quad
  d = \alpha - 1 = 4/3.
  \]
  }
  \item Find $E(X^3)$.

  \texttt{The mgf of $X$ is
  \[
    M_X(t) = (1 - \beta t)^{-\alpha}, \quad t < 1/\beta.
    \]
  From $\log M_X(t) = -\alpha \log(1 - \beta t)$ we have
  \[
  \frac{1}{M} \frac{\diff M}{\diff t} = -\alpha\frac{1}{1 - \beta t}
      (-\beta) = \frac{\alpha\beta}{1 - \beta t}
  \]
  hence
  \[
  \frac{\diff M}{\diff t} = \frac{\alpha\beta}{1 - \beta t}M
  \]
  then
  \[
  \frac{\diff^2 M}{\diff t^2}
  = \alpha\beta\left\{
    \frac{\beta}{(1-\beta t)^2} M
    + \frac{1}{1 - \beta t} \frac{\alpha\beta}{1 - \beta t} M
    \right\}
  = \alpha(1 + \alpha) \beta^2 \frac{1}{(1 - \beta t)^2} M
  \]
  and
  \[
  \frac{\diff^3 M}{\diff t^3}
  = \alpha(1+\alpha) \beta^2 \left\{
    \frac{2\beta}{(1 - \beta t)^3} M
    + \frac{1}{(1 - \beta t)^2} \frac{\alpha\beta}{1 - \beta t} M
  \right\}
  = \frac{\alpha (\alpha+1) (\alpha+2) \beta^3}{(1 - \beta t)^3} M
  \]
  hence
  \[
  E(X^3)
  = \frac{\diff^3 M_X(t)}{\diff t^3} \Bigm|_{t=0}
  = \alpha (\alpha+1) (\alpha+2) \beta^3
  = 910
  \]
  }
  \end{enumerate}

\item If $X$ is a random variable having
pdf $f_X(x) = k x^{3}(1-\frac{x}{12})^4 I(0\le x \le 12)$,
then:
  \begin{enumerate}
  \item Find $k$. (Hint:  Do not expand the expression for $f_X(x)$.
Instead, figure out how this pdf is related to that of a familiar distribution.)

\texttt{%
Looks like $X$ has a beta distribution stretched from $[0,1]$ to
$[0,12]$. We shrink it back by introducing $Y = X/12$, then
\[
f_Y(y) = 12^3 k y^3 (1 - y)^4 I(0 \le y \le 1)
\]
therefore $Y$ has a beta distribution with $\alpha = 4$ and $\beta = 5$.
Then
\[
12^3 k
= \frac{1}{B(\alpha,\beta)}
= \frac{1}{B(4, 5)}
= \frac{\Gamma(9)}{\Gamma(4)\Gamma(5)}
= \frac{8!}{3! 4!}
\]
hence
\[
k = \frac{8!}{3! 4! 12^3} = 35/216
\]
}

  \item Find $E(X)$ and $\var(X)$.

  \texttt{We use the formulas for the mean and variance of
  a beta variable:
  \[
  E(X) = 12 E(Y) = 12\frac{\alpha}{\alpha + \beta} = 16/3,
  \quad
  \var(X)
  = 12^2 \var(Y)
  = 12^2 \frac{\alpha\beta}{(\alpha+\beta)^2 (\alpha + \beta + 1)}
  = 32/9
  \]
  }

  \end{enumerate}

\item For each of the following families, either show why it is
an exponential family or explain why it is not:

  \begin{enumerate}
  \item normal family with $\mu$ known.

  \texttt{%
  \[
  f(x) = \frac{1}{\sqrt{2\pi} \sigma} e^{-(x - \mu)^2/\sigma^2}
  \]
  Let $h(x) = 1$, $c(\theta) = \frac{1}{\sqrt{2\pi}\sigma}$,
    $w_1(\theta) = 1/\sigma^2$, $t_1(x) = -(x - \mu)^2$.
    Yes.
  }

  \item normal family with $\sigma$ known.

  \texttt{%
  \[
  f(x)
  = \frac{1}{\sqrt{2\pi} \sigma} e^{-(x - \mu)^2/\sigma^2}
  = \frac{1}{\sqrt{2\pi}\sigma}
    e^{-\frac{x^2}{\sigma^2} + \frac{2\mu x}{\sigma^2} -
    \frac{\mu^2}{\sigma^2}}
  \]
  Let
    $h(x) = \frac{1}{\sqrt{2\pi}\sigma}$,
    $c(\theta) = 1$,
    $w_1(\theta) = 1$,
    $t_1(x) = -\frac{x^2}{\sigma^2}$,
    $w_2(\theta) = \mu$,
    $t_2(x) = \frac{2x}{\sigma^2}$,
    $w_3(\theta) = -\mu^2/\sigma^2$,
    $t_3(x) = 1$.
    Yes.
  }


  \item family of distributions with
  pdfs $f(x) = \frac{\theta^3}{6}x(\theta-x)I(0 < x < \theta)$,
  where $\theta > 0$.

  \texttt{No.}

  \item gamma family with both $\alpha$ and $\beta$ unknown.

  \texttt{%
  \[
  f(x) = \frac{1}{\Gamma(\alpha) \beta^{\alpha}}
    x^{\alpha - 1} e^{-x/\beta}
    = \frac{1}{\Gamma(\alpha) \beta^{\alpha}}
        e^{(\alpha-1)\log x - 2/\beta}
  \]
  Yes.
  }

\end{enumerate}

\item
Let the random variable $X$ have the pdf
\[
f(x) = \frac{2}{\sqrt{2\pi}}\exp(-x^2/2),
\quad 0 < x < \infty.
\]
\begin{enumerate}
\item
Find the mean and variance of $X$.
(This distribution is sometimes called a {\em{folded normal}}.)

\texttt{%
\[\begin{split}
E(X)
&= \int_0^{\infty} \frac{2}{\sqrt{2\pi}} x \exp(-x^2/2) \diff x
\\
&= \int_0^{\infty} \frac{2}{\sqrt{2\pi}} \exp(-x^2/2) \diff (x^2/2)
\\
&= -\frac{2}{\sqrt{2\pi}} \exp(-x^2/2) \Bigm|_{x=0}^{\infty}
\\
&= \sqrt{\frac{2}{\pi}}
\end{split}
\]
\[\begin{split}
E(X^2)
&= \int_0^{\infty} \frac{2}{\sqrt{2\pi}} x^2 \exp(-x^2/2) \diff x
\\
&= - \int_0^{\infty} \frac{2}{\sqrt{2\pi}} x \diff \exp(-x^2/2)
\\
&= - \frac{2}{\sqrt{2\pi}} x \exp(-x^2/2) \biggm|_{x=0}^{\infty}
 + \int_0^{\infty} \frac{2}{\sqrt{2\pi}} \exp(-x^2/2) \diff x
\\
&= 1
\end{split}
\]
\[
\var(X) = 1 - \frac{2}{\pi}
\]
}

\item If $X$ has the folded normal distribution,
find the transformation $g(X)=Y$ and values of
$\alpha$ and $\beta$ so that $Y \sim \text{Gamma}(\alpha,\beta)$.

\texttt{%
In a gamma distribution,
the $X$ in the exponent should have power 1.
This suggests we can try $Y = X^2$, $Y > 0$.
Then
\[
f_Y(y)
= \frac{1}{\sqrt{2\pi}} y^{-1/2} \exp(-y/2)
\]
This is a gamma density with
$\alpha = 1/2$ and $\beta = 2$.
Check the normalizing constant is correct,
noticing $\Gamma(1/2) = \sqrt{\pi}$.
}
\end{enumerate}

\item 4.4 on page 192.
\[
f(x,y) = \begin{cases}
    C(x + 2y) & \text{if $0 < y < 1$ and $0 < x < 2$}\\
    0 & \text{otherwise}
\end{cases}
\]

\begin{enumerate}
\item
\[
\int_0^2 \int_0^1 C(x + 2y) \diff y \diff x
= 4C = 1
\quad\Rightarrow\quad
C = 1/4
\]
\item
\[
f_X(x)
= \int_0^1 \frac{1}{4}(x + 2y) \diff y
= \frac{1}{4}x + \frac{1}{4}
\]
\item
\[
F(x,y)
=
\begin{cases}
 0,  & x<0\text{ or } y<0\\
 \frac{1}{8}(2xy^2 + x^2y),    & 0 \le x < 2,\, 0 \le y < 1\\
 \frac{1}{2}(y+y^2),    & 2 < x,\, 0 \le y < 1\\
 \frac{1}{8}(2x +x^2),    & 0 \le x < 2,\, 1 < y\\
 1,   & 2 \le x,\, 1 \le y
\end{cases}
\]

\item
$z = g(x) = 9/(x + 1)^2$,
$1 < z < 9$.
$x = g^{-1}(z) = 3z^{-1/2} - 1$.
\[
f_Z(z)
= \left(
    \frac{1}{4}(3z^{-1/2} - 1) + \frac{1}{4}
    \right)
    (3/2)z^{-3/2}
= \frac{9}{8}z^{-2}
,\;
1 < z < 9.
\]

\end{enumerate}

\end{enumerate}

\end{document}
