\documentclass[12pt]{article}
\usepackage{techart}
\begin{document}
\title{STAT 651 HW 1 Solutions}
\maketitle

\begin{enumerate}

\item (15 points)
\begin{enumerate}
\item
Let H and T denote ``head'' and ``tail'', respectively.

$S = \bigl\{$%
HHHH, HHHT, HHTH, HHTT, HTHH, HTHT, HTTH, HTTT,
THHH, THHT, THTH, THTT, TTHH, TTHT, TTTH, TTTT%
$\bigr\}$.

Finite.

\item
$S = \{0,1,2,\dotsc\}$.
Countably infinite.
(If you say there has to be an upper limit on the number of leaves a
plant can have, it's not totally unreasonable. Then it's finite.
In doing a math model, such an upper limit is usually unnecessary
and it often complicates the model to some degree without benefit.)

\item
$S = \{0, 1, 2,\dotsc,\}$. Countably infinite.

Or
$S = (0,\infty)$. Uncountably infinite.

\item
$S = (0, \infty)$. Uncountably infinite.
(If you close the interval at a specific, safe, upper limit,
the sample space is still uncountable.)

\item
$S = [0,1]$. Uncountably infinite.

Or $S = \{x \in Q:\; 0 \le x \le 1\}$
where $Q$ is the set of rational numbers.

\end{enumerate}

\item (12 points)
\begin{enumerate}
\item $P(A) + P(B) - P(A\cap B)$
\item $P(A) + P(B) - 2P(A\cap B)$
\item $P(A) + P(B) - P(A\cap B)$
\item $1 - P(A\cap B)$.
\end{enumerate}

\item (12 points)
\begin{enumerate}
\item
This is the event that an US birth results in twin females
who are from one single egg.

\item
\[
P(A\cap B \cap C)
= P(C)\, P(B\given C)\, P(A\given B, C)
= \frac{1}{90}\, \frac{1}{3}\, \frac{1}{2}
= \frac{1}{540}\quad (=0.001852)
\]
\end{enumerate}

\item
(15 points)

\begin{enumerate}

\item
Use 1, 2, 3, 4, 5, 6 to designate the wireless plans
and 'U', 'O' to mean 'under' and 'over', respectively.
$
S = \{1U, 1O, 2U, 2O, 3U, 3O, 4U, 4O, 5U, 5O, 6U, 6O\}
$,
where `U' means `under', `O' means `over'.

\item
$ B^c = \{4U, 4O, 5U, 5O, 6U, 6O\} $

$ A \cup B = \{1U, 1O, 2U, 2O, 3U, 3O, 4O, 5O, 6O\} $

$ C \cap D = \{6O\} $

$ (A \cap D)^c = \{1U, 1O, 2U, 3U, 3O, 4U, 5U, 5O, 6U\} $
\end{enumerate}


\item
(12 points)

\begin{enumerate}

\item
$P(\text{at least two})
 = 1 - P(\text{at most one})
 = 1 - .428
 = .572$.

\item
Take the combination of purchases of the five customers as the outcome,
then the events ``all gas'', ``all electric'',
``both types'' form a partition of the sample space.
Therefore
\[
P(\text{both types})
= 1 - P(\text{all gas}) - P(\text{all electric})
= 1 - .116 - .005
= .879\]
\end{enumerate}


\item
(12 points)

\begin{enumerate}

\item
\[ {20\choose 7} = \frac{20!}{7! 13!} =
\frac{20\times\dotsb\times 14}{7\times\dotsb\times1} = 77520\]

\item
Solution 1:
$ {10 \choose 7} \bigm/ {20 \choose 7}
  = 1/646 = 0.0015$.

Solution 2:
$ \frac{10}{20} \frac{9}{19} \frac{8}{18} \frac{7}{17} \frac{6}{16}
\frac{5}{15} \frac{4}{14} = 1/646 = 0.0015$.
\end{enumerate}

\item (25 points)

Denote the number of outcomes in the sample space
by $N_S$ and the number of elements in the sigma algebra by $N_B$.
\begin{enumerate}
\item
$N_S = 2^4 = 16$,
$N_B = 2^{N_S} = 65536$.

\item
$N_S = 5$,
$N_B = 2^{N_S} = 32$.

\item
$N_S = 3 \times 3 = 9$,
$N_B = 2^{N_S} = 512$.

\item
$N_S = 2^4 = 16$,
$N_B = 2^{N_S} = 65536$.

\item
$N_S = (n_1 + 1) (n_2 + 1)$,
$N_B = 2^{N_S} = 2^{(n_1 + 1)(n_2 + 1)}$.
\end{enumerate}

\end{enumerate}
\end{document}

