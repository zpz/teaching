\documentclass[12pt]{article}
\usepackage{techart}
\begin{document}
\title{STAT 651 HW 9}
\author{Due Friday Dec 3}
\date{}
\maketitle

\begin{enumerate}
\item
  Let $f(x,y) = 2\exp(-(x+y))I(0 < x < y < \infty))$.  Let $U = X$ and $V=X+Y$.
  \begin{enumerate}
  \item Find and sketch the support for $(X,Y)$.
  \item Are $X$ and $Y$ independent?  Explain.
  \item Find and sketch the support for $(U,V)$.
  \item Find $f_{U,V}(u,v)$, the joint density for $(U,V)$.
  \item Are $U$ and $V$ independent?  Find the marginal distributions, $f_U(u)$ and $f_V(v)$.  (Be sure to state the range of possible values for $U$ and $V$.)
  \end{enumerate}

\item
  Let $f(x,y) = k(x+y)$, where $0 < x,y < 1$ and $y<x$.
  Let $U = X+Y$ and $V = 2X-Y$.
  \begin{enumerate}
  \item Find $k$ so that $f(x,y)$ is a density function.
  \item Find and sketch the support for $(X,Y)$.
  \item Find and sketch the support for $(U,V)$.
  \item Find $f_{U,V}(u,v)$, the joint density for $(U,V)$.
  \item Are $U$ and $V$ independent?  Explain.
  \end{enumerate}

\item Use mgfs to find the distribution of $X+Y$, where:
\begin{enumerate}
\item $X \sim \text{Gamma}(\alpha_X,\beta)$ and
    $Y \sim \text{Gamma}(\alpha_Y,\beta)$ and $X$ and $Y$ are independent.
\item $X \sim \text{Poisson}(\theta)$ and $Y \sim \text{Poisson}(\lambda)$
    and $X$ and $Y$ are independent.
\end{enumerate}

\item More practice with joint and marginal distributions.

\begin{enumerate}
\item ``Low correlation'' example.
Let $A$ be the region in $R^2$ bounded by the lines $x=0$, $x=1$, $y=2-x$ and $y=1-x$.
Let $f(x,y) = 1$, $(x,y) \in A$, be the joint density of $(X,Y)$.

\begin{enumerate}
\item Show that $X \sim \text{unif}(0,1)$ and $E(X)=1/2$, $V(X)=1/12$.
\item Show that $E(Y)=1$, $E(Y^2)=7/6$ so that $V(Y)=1/6.$
\item Show that $E(XY) = 5/12.$
\item Show that, consequently,
    $\cov(X,Y) \approx -.08333$ and
    $\operatorname{cor}(X,Y)\approx -.7071$.
\end{enumerate}

\item ``High correlation'' example.
Let $B$ be the region in $R^2$ bounded by the lines $x=0$, $x=1$, $y=1.6-x$ and $y=1.4-x$.
Let $f(x,y) = 5$, $x \in B$, be the joint density of $(X,Y)$.

\begin{enumerate}
\item Show that $X \sim \text{unif}(0,1)$ so that $E(X)=1/2$ and
    $V(X)=1/12$.
\item Show that $E(Y)=1$, $E(Y^2) = 1.086667$
    so that $V(Y)\approx .086667$.
\item Show that $E(XY) \approx .416667$.
\item Show that, consequently,
    $\cov(X,Y)\approx -.08333$  and $\operatorname{cor}(X,Y)\approx -.9806$.
\end{enumerate}

\end{enumerate}

\item
Uncorrelated random variables are not necessarily independent random variables.
Let $X \sim N(0,1)$ and define $Y = X^2$.
Show that $E(XY) = E(X)E(Y)$,
and subsequently $X$ and $Y$ are uncorrelated.
Also show that $X$ and $Y$ are dependent.
(Hint:  This is not like the problems you've done with joint distributions.
Do not try to find the joint distribution of $(X,Y)$.
Instead, do all your calculations in terms of $X$ and its distribution.
It's okay to explain the lack of independence informally.)

\item \textbf{Hierarchical models.}
Consider the following Las Vegas two-stage casino prize selection method.
A hat has 13 cards in it, numbered from 1--13.  Draw a card at random; let $N$ be the number
on the card.
You then draw $N$ many chocolate chip cookies independently
from an  infinite cookie jar, where the individual
cookies have a variable number of chocolate chips on them.
The number of chocolate chips on the $i$th cookie is
$Y_i \stackrel{\text{iid}}{\sim} \text{Poisson}(10)$.
For example,
if you draw the number 7, you then extract 7 cookies
from a jar in which the cookies have, on average,
10 chocolate chips apiece.
Let $Z = \sum_{i=1}^N Y_i$ be the
total number of chocolate chips on your $N$ many cookies.
Find $E(Z)$ and $V(Z)$, the mean and variance
of the number of chocolate chips in your prize.


\item
Show that $\cov(X,c)=0$ for any random variable $X$ with finite mean and any constant $c$.

\item
Let $X_1$, $X_2$ and $X_3$ be uncorrelated random variables, each with mean $\mu$
and variance $\sigma^2$.  Find, in terms of $\mu$ and $\sigma^2$, the quantities
$\cov(X_1+X_2,X_2+X_3)$ and $\cov(X_1+X_2,X_1-X_2)$.

\end{enumerate}

\end{document}
