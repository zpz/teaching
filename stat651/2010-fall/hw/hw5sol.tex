\documentclass[12pt]{article}
\usepackage{techart}
\begin{document}
\title{STAT 651 HW 5 Solution}
\maketitle

\begin{enumerate}
\item (24 pts)

\begin{align*}
E(X) &= \int_0^1 x \diff x = \frac{1}{2}\int_0^1 \diff x^2 = 1/2
\\
E(X^2) &= \int_0^1 x^2 \diff x = \frac{1}{3} \diff x^3 = 1/3
\\
\var(X) &= E(X^2) - \bigl(E(X)\bigr)^2 = 1/12
\\
E(\sin \pi X)
&= \int_0^1 \sin\pi x \diff x
= - \frac{1}{\pi} \int_0^1 \diff \cos\pi x
= -\frac{1}{\pi} (-1 - 1)
= 2/\pi
\\
E(X\sin\pi X)
&= \int_0^1 x\sin\pi x \diff x
\\
&= -\frac{1}{\pi}\int_0^1 x \diff \cos\pi x
\\
&= -\frac{1}{\pi}\biggl(
    x\cos\pi x\Bigm|_0^1 - \int_0^1 \cos\pi x \diff x
        \biggr)
\\
&= \frac{1}{\pi} + \frac{1}{\pi^2} \int_0^1 \diff \sin \pi x
\\
&= \frac{1}{\pi}
\\
&= 1/\pi
\\
E(\cos\pi X)
&= \int_0^1 \cos\pi x\diff x
= \frac{1}{\pi}\int_0^1 \diff \sin\pi x
= 0
\\
E(X\cos\pi X)
&= \int_0^1 x\cos\pi x\diff x
\\
&= \frac{1}{\pi} \int_0^1 x \diff \sin\pi x
\\
&= \frac{1}{\pi} \biggl(
    x\sin\pi x\Bigm|_0^1 - \int_0^1 \sin\pi x\diff x\biggr)
\\
&= \frac{1}{\pi^2} \int_0^1 \diff \cos\pi x
\\
&= -2/\pi^2
\end{align*}

\item (8 pts)

$y = g(x) = \sigma x + \mu$,
$x = g^{-1}(y) = \frac{y - \mu}{\sigma}$,
$\frac{\diff g^{-1}(y)}{\diff y} = 1/\sigma$
\[
f_Y(y)
= \frac{1}{\sqrt{2\pi}}
    \exp\Bigl\{-\frac{1}{2}\frac{(y - \mu)^2}{\sigma^2}\Bigr\}
    \cdot
    \frac{1}{\sigma}
= \frac{1}{\sqrt{2\pi}\, \sigma}
    \exp\Bigl\{-\frac{1}{2}\frac{(y - \mu)^2}{\sigma^2}\Bigr\}
\]

\item (8 pts)

\[
\var(aX)
= E\bigl[\bigl(aX - E(aX)\bigr)^2\big]
= E\bigl[a^2\bigl(X - E(X)\bigr)^2\big]
= a^2 E\bigl[\bigl(X - E(X)\bigr)^2\big]
= a^2\var(X)
\]
\[
\var(X + c)
= E\bigl[\bigl(X + c - E(X + c)\bigr)^2\bigr]
= E\bigl[\bigl(X - E(X)\bigr)^2\bigr]
= \var{X}
\]
\[
\var(aX + c)
= \var(aX)
= a^2\var(X)
\]

Note: we can not use properties of variance such as
$\var(X) = E(X^2) - E^2(X)$,
because this is more ``advanced'' properties that those
we are asked to prove.
However, we can use basic properties of the $E$ operator.

\item (16 pts)

\begin{enumerate}
\item
\[
M_X(t)
= \sum_{x=0}^{\infty} e^{tx} \frac{e^{-\lambda} \lambda^x}{x!}
= \sum_{x=0}^{\infty} \frac{e^{-\lambda} (\lambda e^t)^x}{x!}
= e^{\lambda(e^t - 1)} \sum_{x=0}^{\infty} \frac{e^{-\lambda e^t} (\lambda e^t)^x}{x!}
= e^{\lambda (e^t - 1)}
\]
\item
\[
E(X)
= \frac{\diff e^{\lambda(e^t - 1)}}{\diff t} \biggm|_{t=0}
= e^{\lambda(e^t - 1)} \lambda e^t \biggm|_{t=0}
= \lambda
\]
\[
E(X^2)
= \frac{\diff^2 e^{\lambda(e^t - 1)}}{\diff t^2} \biggm|_{t=0}
= \frac{\diff e^{\lambda(e^t - 1)} \lambda e^t}{\diff t} \biggm|_{t=0}
= \lambda e^{\lambda e^t - \lambda + t} (\lambda e^t + 1) \bigm|_{t=0}
= \lambda (\lambda + 1)
\]
\[
\var(X)
= \lambda(\lambda + 1) - \lambda^2 = \lambda
\]
(When finding the 2nd derivative,
make use the 1st derivative just obtained.)
\end{enumerate}


\item (16 pts)

\begin{enumerate}
\item
\[
M_X(t)
= \int_{-\infty}^{\infty}
    e^{tx}
    \frac{1}{\sqrt{2\pi}}
    e^{-x^2/2}
    \diff x
= e^{t^2/2} \int_{-\infty}^{\infty} \frac{1}{\sqrt{2\pi}}
    e^{-(x + t)^2 / 2}
    \diff x
= e^{t^2/2}
\]
\item
\[
E(X)
= \frac{\diff e^{t^2/2}}{\diff t} \Bigm|_{t=0}
= e^{t^2/2}\, t \bigm|_{t=0}
= 0
\]
\[
E(X^2)
= \frac{\diff^2 e^{t^2/2}}{\diff t^2} \biggm|_{t=0}
= \bigl(e^{t^2/2} + t^2 e^{t^2/2}\bigr) \bigm|_{t=0}
= 1
\]
\[
\var(X) = 1 - 0^2 = 1
\]
\end{enumerate}

\item (8 pts)

\[
M_Y(t)
= E\bigl(e^{tY}\bigr)
= E\bigl(e^{\sigma t X + t\mu}\bigr)
= E\bigl(e^{(\sigma t)X} e^{t\mu}\bigr)
= e^{t\mu} M_X(\sigma t)
= e^{\mu t} e^{\sigma^2 t^2/2}
\]
(Note:
since this is a function of $t$, write $\mu t$ instead of $t\mu$.)

\item (8 pts)

Let $Y = -X$, then
$X = g^{-1}(Y) = -Y$.
\[
f_Y(y) = f_X(-y) |-1| = f_X(-y) = f_X(y)
\]
because $f_X(x)$ is an even function.
This shows that $X$ and $Y$ have the same pdf, hence the same
distribution.
Before this conclusion,
it should also be noted that $X$ and $Y$ have the same support,
because $f_X(x)$ is even (suggesting the support of $X$ is symmetric
about 0) and $Y = -X$.

\item (16 pts)

\begin{enumerate}
\item Normal. Gamma (including exponential).
\item Uniform. Or make up a pdf that has a flat segment and slopes down
on both sides.
\item This is an exponential pdf. Its mode is 0.
It's easy to show that 0 is the only value that satisfies the definition
of mode.
\end{enumerate}
\end{enumerate}

\end{document}
