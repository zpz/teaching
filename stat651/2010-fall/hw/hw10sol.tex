\documentclass[12pt]{article}
\usepackage{techart}

\begin{document}
\title{STAT 651 HW 10 Solution}
\maketitle


\begin{enumerate}
\item Let $X$ and $Y$ be independent uniform(0,1) RV's. Find
$P(X/Y\le t)$ and $P(XY\le t)$. (Pictures will help.)

\hrulefill

If $t \le 0$, the probabilities are both 0.
So we'll discuss for $t > 0$ only.
\begin{enumerate}
\item $P(X/Y \le t$, ie $P(Y \ge X/t)$.

If $t \ge 1$,
\[
P(Y \ge X/t)
= 1 - (1)(1/t)/2 = 1 - 1/(2t)
\]
If $t < 1$,
\[
P(Y \ge X/t)
= (1)(t)/2
= t/2
\]

\item $P(XY \le t)$, ie $(Y \le t/X)$.

If $t \ge 1$, $P(Y \le t/X) = 1$.

If $t < 1$,
\[
P(Y \le t/X)
= 1 - \int_t^1\int_{t/x}^1 1 \diff y \diff x
= 1 - t + t\log t
\]

\end{enumerate}

\item Let $(X,Y)$ follow a multivariate normal distribution with parameters
$\mu_X = 5$, $\mu_Y = -1$, $\sigma_X = 1$, $\sigma_Y=4$ and $\rho$ (to be specified momentarily).
(You can use the matrix formulas learned in class.)

  \begin{enumerate}
  \item Find the conditional distribution of $Y$ given $X=7$, assuming $\rho = 0.6$.
  Is the conditional variance of $Y$ greater than or less than the unconditional (marginal) variance?
  \item Find the conditional distribution of $Y$ given $X=7$, assuming $\rho = -0.6$.
  Is the conditional variance of $Y$ greater than or less than the unconditional (marginal) variance?
  \end{enumerate}

\hrulefill

\begin{enumerate}
\item
$\rho = .6 \Rightarrow \cov(X,Y) = .6 \times 1 \times 4 = 2.4$, hence
\[
\begin{bmatrix}X\\ Y\end{bmatrix}
\sim N\biggl(
    \begin{bmatrix}5\\ -1\end{bmatrix},\;
    \begin{bmatrix}1 & 2.4\\ 2.4 & 16\end{bmatrix}
    \biggr)
\]
$Y \given X = 7$ is normal with mean
\[
\mu_Y + \cov(Y,X) \cov(X,X)^{-1} (\mu_X - x)
= -1 + 2.4 \times (1/1) \times (7 - 5)
= 3.8
\]
and variance
\[
\cov(Y) - \cov(Y,X) \cov(X,X)^{-1} \cov(X,Y)
= 16 - 2.4 \times (1/1) \times 2.4
= 10.24
\]
The conditional variance is smaller than the unconditional one.

\item
$\rho = -.6 \Rightarrow \cov(X,Y) = -.6 \times 1 \times 4 = -2.4$, hence
\[
\begin{bmatrix}X\\ Y\end{bmatrix}
\sim N\biggl(
    \begin{bmatrix}5\\ -1\end{bmatrix},\;
    \begin{bmatrix}1 & -2.4\\ -2.4 & 16\end{bmatrix}
    \biggr)
\]
$Y \given X = 7$ is normal with mean
\[
\mu_Y + \cov(Y,X) \cov(X,X)^{-1} (\mu_X - x)
= -1 + (-2.4) \times (1/1) \times (7 - 5)
= -5.8
\]
and variance
\[
\cov(Y) - \cov(Y,X) \cov(X,X)^{-1} \cov(X,Y)
= 16 - (-2.4) \times (1/1) \times (-2.4)
= 10.24
\]
The conditional variance is smaller than the unconditional one.
\end{enumerate}

\item Let $X$, $Y$ and $Z$ be independent Poisson random variables with parameters
$\lambda_X$, $\lambda_Y$ and $\lambda_Z$ respectively. Let $U = X+Z$ and $V = Y+Z$.
Find $\cov(U,V)$ and $\operatorname{cor}(U,V)$.

\hrulefill

\[
\cov(U,V)
= \cov(X,Y) + \cov(X,Z) + \cov(Z,Y) + \cov(Z,Z)
= 0 + 0 + 0 + \lambda_Z
= \lambda_Z
\]
$\var(U) = \var(X) + \var(Y) = \lambda_X + \lambda_Y$,
$\var(V) = \lambda_Y + \lambda_Z$.
\[
\operatorname{cor}(U,V)
= \cov(U,V) / \sqrt{\var(U)\var(V)}
= \lambda_Z / \sqrt{(\lambda_X + \lambda_Y)(\lambda_Y + \lambda_Z)}
\]

\item Let $\vec{X} \sim \text{Multinomial}(7; .4,.3,.2,.1)$.  For
example, $\vec{X}$ represents
the number of red, green, blue, and white balls drawn when drawing samples of size 7
with replacement from an urn containing
4 red, 3 green, 2 blue, and 1 white balls.  Let $f(\vec{x})$ be the corresponding pmf.
  \begin{enumerate}
  \item Find $f(3,2,1,1)$.
  \item Find $f(1,1,4,1)$. (So more blue balls than there are in the urn.)
  \item Find $f(0,3,2,2)$.
  \item (Optional) For how many vectors of $\vec{x}$ is $f(\vec{x})$ non-zero?
  \end{enumerate}

\hrulefill

  \begin{enumerate}
  \item
  \[
  f(3,2,1,1) = \frac{7!}{3!2!1!1!}.4^3 .3^2 .2^1 .1^1 = 0.0484
  \]
  \item
  \[
  f(1,1,4,1) = \frac{7!}{1!1!4!1!}.4^1 .3^1 .2^4 .1^1 = 0.004
  \]
  \item
  \[
  f(0,3,2,2) = \frac{7!}{0!3!2!2!}.4^0 .3^3 .2^2 .1^2 = 0.0023
  \]
  \item
  This is the number of combinations for 4 non-negative integers that
  sum to 7.
  Imagine throwing 7 balls into 4 baskets;
  the number of possible outcomes is this number.
  This is the ``unordered, with replacement'' situation in the table on
  page~16.
  \[
  {n + r - 1 \choose r}
  = {4 + 7 - 1\choose 7}
  = 120
  \]
  \end{enumerate}

\item Jensen's Inequality states that if $g(x)$ is convex on the support of $X$,
 then $E(g(X)) \ge g(E(X)).$
  \begin{enumerate}
  \item Show that if $X \sim N(-1,\sigma^2)$, then $E(|X|)\ge 1.$
  What bound would you get if $X \sim N(-10,\sigma^2)$?  What if $X \sim N(3,\sigma^2)$?
  \item Show that if $X \sim $Poisson($\lambda$) then $E(X^3) \ge \lambda^3.$ (Can you find the
exact value of $E(X^3)$?)
  \item Use Jensen's Inequality to show that if $X \sim N(0,\sigma^2)$,
   then $E(X^4) \ge (\sigma^2)^2 = \sigma^4.$
  \end{enumerate}

\hrulefill

  \begin{enumerate}
  \item
    $g(x) = |x|$ is convex, hence
    if $X \sim N(-1,\sigma^2)$ then
    $E|X| \ge |E(X)| = |-1| = 1$.
    If $X \sim N(-10,\sigma^2)$ then
    $E|X| \ge |E(X)| = |-10| = 10$.
    If $X \sim N(3,\sigma^2)$ then
    $E|X| \ge |E(X)| = |3| = 3$.

    \item
    $g(x) = x^3$, $x\ge 0$, is convex.
    Hence
    $E(X^3) \ge \bigl(E(X)\bigr)^3 = \lambda^3$.

    $E(X^3)$ can be found by the Poisson mgf,
    $M_X(t) = e^{\lambda(e^2-1)}$.
    \[
    \frac{\diff^3 M_X(t)}{\diff t^3} \biggm|_{t=0}
    = \lambda \bigl((1 + \lambda e^t)^2 + \lambda e^t\bigr)
        e^{t + \lambda(e^t -1)} \biggm|_{t=0}
    = \lambda(1 + 3\lambda + \lambda^2)
    \]

    \item
    $x^2$ is a convex function.
    \[
    E(X^4) \ge \bigl(E(X^2)\bigr)^2 = (\sigma^2)^2 = \sigma^4
    \]
  \end{enumerate}

\item Let $X_i \stackrel{iid}{\sim} $ Exp($\beta$), $i=1,2,\ldots,n$.
Find the distribution of $X_{(1)}$, the first order statistic, and
of $X_{(n)}$, the $n$-th order statistic.  Do either of these correspond to a familiar distribution?

\hrulefill

The pdf is $f_{X_1}(x) = \frac{1}{\beta} e^{-x/\beta}$, $x > 0$.
The cdf is $F_{X_1}(x) = 1 - e^{-x/\beta}$, $x > 0$.

\[
F_{X_{(1)}}(x)
= 1 - P(X_{(1)} \ge x)
= 1 - \bigl(P(X_1 \ge x)\bigr)^n
= 1 - \bigl(e^{-x/\beta}\bigr)^n
= 1 - e^{-x/(\beta/n)}
\]
This is $\text{Exp}(\beta/n)$.
\[
F_{X_{(n)}}(x)
= P(X_{(n)} \le x)
= \bigl(P(X_1 \le x)\bigr)^n
= \bigl(1 - e^{-x/\beta}\bigr)^n
\]
I don't know whether this is a familiar distribution.
Taking derivative (to get the pdf) or using simple transformations do
not appear to help.

\item
Let $X_i \stackrel{iid}{\sim} N(\mu_X,\sigma^2)$, $i=1,2,\ldots,n$ and
$Y_j \stackrel{iid}{\sim} N(\mu_Y,\sigma^2)$, $j = 1,\ldots,m$.
Assume that the $\{X_i\}$ and $\{Y_j\}$ are independent.
     \begin{enumerate}
     \item Find the sampling distribution of $\bar{X}-\bar{Y}$.
     \item Find the distribution of $\frac{1}{\sigma^2}[(n-1)S_X^2 + (m-1)S_Y^2]$.
     \end{enumerate}

\hrulefill

\begin{enumerate}
\item
First we know
$\overline{X} \sim N(\mu_X, \sigma^2/n$ and
$\overline{Y} \sim N(\mu_Y, \sigma^2/m)$.
Because $\{X_i\}$ and $\{Y_i\}$ are independent,
the two sample means are independent.
Their difference is normal with mean
$\mu_X - \mu_Y$ and variance $\sigma^2(1/n + 1/m)$.

\item
Because
$(n-1)S^2_{X}/\sigma^2 \sim \chi^2_{n-1}$,
$(m-1)S^2_{Y}/\sigma^2 \sim \chi^2_{m-1}$,
and these two $\chi^2$ variables are independent,
their sum is a $\chi^2_{n+m-2}$ variable.
\end{enumerate}

\item Let $X_i \stackrel{iid}{\sim} N(0,1)$, $i=1,\ldots,5.$
Let $X_6$ be another independent observation, also $N(0,1)$.
Let $\bar{X} = (X_1 + X_2 + \cdots + X_5)/5$, i.e.\ the mean of just the first 5 $X$'s.
 What is the distribution of
\begin{enumerate}
\item $W = \sum_1^5 X_i^2$? Why?
\item $U = \sum_1^5 (X_i-\bar{X})^2$? Why?
\item $\sum_1^5 (X_i-\bar{X})^2+X_6^2$?  Why?
\item $\sqrt{5}X_6/\sqrt{W}$? Why?
\item $2X_6/\sqrt{U}$? Why?
\item $2(5\bar{X}^2 + X_6^2)/U$? Why?
\end{enumerate}

\hrulefill

\begin{enumerate}
\item $\chi^2_5$.
\item $\chi^2_4$.
\item $\chi^2_5$.
\item $t_5$.
\item $t_4$.
\item $F_{2,4}$.
\end{enumerate}

\end{enumerate}

\end{document}
