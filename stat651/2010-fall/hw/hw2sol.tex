\documentclass[12pt]{article}
\usepackage{techart}
\begin{document}
\title{STAT 651 HW 2 Solutions}
\maketitle

\begin{enumerate}
\item
    \begin{enumerate}
    \item
    \[
        {13\choose 1}{4\choose 3}\times {12\choose 1}{4\choose 2}
        = 3744
    \]
    Note: one ``kind'' gets 3 cards and the other gets 2,
    therefore these two ``kinds'' are different.
    We can imagine finishing the task in two steps:
    the first step does the 3-card kind,
    and after that there are ${12\choose 1}$ choices for the
    2-card kind.
    \item
    \[
        {13\choose 2}{4\choose 2}^2\times {44\choose 1}
        = 123552
    \]
    Note: there are two special kinds, each getting two cards.
    But there is no ordering between these two ``kinds''.
    Different from the one above.
    \item
    \[
        {13\choose 5}{4\choose 1}^5 = 1317888
    \]
    \end{enumerate}

\item
Let $E_i$, $i=1,2,\dotsc$, be the event that $A$ wins at her $i$-th toss,
meaning $A$ and $B$ both get all tails in their previous tosses,
and $A$ gets head in the $i$-th toss.
\begin{enumerate}
\item
We have
$P(E_i) = (1/2)^{i-1}\times (1/2)^{i-1} \times .5$.
(Check this formula is OK for $i=1$.)
The probability that $A$ wins is
\[
\sum_{i=1}^{\infty} P(E_i)
= \sum_{i=1}^{\infty} \Bigl(\frac{1}{4}\Bigr)^{i-1} \times \frac{1}{2}
= \frac{1}{2} \times \frac{1}{1 - \frac{1}{4}}
= 2/3
\]
where we have used the formula~(1.5.4) on page~31.

\item
Now we have
$P(E_i) = (1-p)^{i-1}\times (1-p)^{i-1} \times p$.
Then the probability that $A$ wins is
\[
\sum_{i=1}^{\infty} P(E_i)
= \sum_{i=1}^{\infty} \bigl((1-p)^2\bigr)^{i-1}\times p
= p \times \frac{1}{1 - (1-p)^2}
= \frac{1}{2-p}
\]

\item
From the above,
for all $p \in (0,1)$,
\[
P = \frac{1}{2 - p} > \frac{1}{2 - 1} = \frac{1}{2}
\]
\end{enumerate}

\item
Let
$M$, $F$, $B$ be the events that a person
is male, is female, and is color-blind, respectively.
Then
\[\begin{split}
P(M \given B)
= \frac{P(M \cap B)}{P(B)}
&= \frac{P(M \cap B)}{P(B\cap M) + P(B\cap F)}
\\
&= \biggl(1 + \frac{P(B\cap F)}{P(M\cap B}\biggr)^{-1}
\\
&= \biggl(1 + \frac{P(F)\,P(B\given F)}{P(M)\,P(B\given M)}\biggr)^{-1}
\\
&= \biggl(1 + \frac{.5\times .25}{.5\times 5}\biggr)^{-1}
\\
&= 0.9524
\end{split}
\]

\item
Denote the events of
hitting the target $k$ times by
$H_k$ and at least $k$ times by
$H_{k+}$. Then
\[
P(H_k) = {10\choose k}\Bigl(\frac{1}{5}\Bigr)^k
    \Bigl(\frac{4}{5}\Bigr)^{10-k}
\]

\begin{enumerate}
\item
\[
P(H_{2+})
= 1 - P(H_0) - P(H_1)
= 1 - \Bigl(\frac{4}{5}\Bigr)^{10}
    - 10\times\frac{1}{5} \Bigl(\frac{4}{5}\Bigr)^9
= .6242
\]

\item
\[
P(H_{2+}\given H_{1+})
= \frac{P(H_{2+})}{1 - P(H_0)}
= \frac{.6242}{1 - \bigl(\frac{4}{5}\bigr)^{10}}
= .6993
\]
\end{enumerate}

\item
\begin{enumerate}
\item
\[
P(A\given B)
= \frac{P(A\cap B)}{P(B)}
= P(A\cap B)
= P(A) + P(B) - P(A\cup B)
= P(A) + 1 - P(B)
= P(A)
\]
where we have used
$P(B) = 1$,
and $P(A\cup B) \ge P(B) = 1$,
hence $P(A\cup B) = 1$.

\item
\[
P(B\given A)
= \frac{P(B\cap A)}{P(A)}
= \frac{P(A)}{P(A)}
= 1
\]
\[
P(A\given B)
= \frac{P(A\cap B)}{P(B)}
= \frac{P(A)}{P(B)}
\]
where we have used $B\cap A = A$ because $A\subset B$.

\item
\[
P(A\given A\cup B)
= \frac{P\big(A\cap (A\cup B)\bigr)}{P(A\cup B)}
= \frac{P(A)}{P(A\cup B)}
= \frac{P(A)}{P(A) + P(B)}
\]
where
$P(A\cup B) = P(A) + P(B)$ is due to the fact
that $A$ and $B$ are mutually exclusive
\end{enumerate}

\item
\begin{enumerate}
\item
\[
P(A\given B)
= \frac{P(A\cap B)}{P(B)}
= \frac{P(\emptyset)}{P(B)}
= 0
\ne P(A)
\]
(where we have used the fact that
$A$ and $B$ are mutually exclusive)
hence
$A$ and $B$ are not independent.

\item
\[
P(A\cap B)
= P(A)\, P(B\given A)
= P(A)\, P(B)
> 0
\]
(where we have used the fact that
$A$ and $B$ are independent)
hence
$A\cap B \ne \emptyset$,
that is,
$A$ and $B$ are not mutually exclusive.
\end{enumerate}

\item
Let $H_k$, $k=0,1,2,3,4$,
be the event of having $k$ heads in the outcome.
Then
\[
P(H_k)
= {4\choose k} (1/2)^k (1/2)^{4-k}
= \frac{3}{2} \frac{1}{k!\,(4-k)!}
\]

\begin{center}
\begin{tabular}{ccc}
$k$ & $P(H_k)$ & $X$ \\ \hline
0   & $\frac{1}{16}$  & 0 \\[5pt]
1   & $\frac{1}{4}$  & 3 \\[5pt]
2   & $3\over 8$  & 4 \\[5pt]
3   & $1\over 4$  & 3 \\[5pt]
4   & $1\over 16$  & 0 \\ \hline
\end{tabular}
\end{center}

Therefore the pmf of $X$ is
\[
P(x) = \begin{cases}
    \frac{1}{8},   & x = 0\\
    \frac{1}{2},   & x = 3\\
    \frac{3}{8},   & x = 4\\
    0,             & \text{otherwise}
    \end{cases}
\]

The cdf of $X$ is
\[
F(x) = \begin{cases}
    0,            & x < 0\\
    \frac{1}{8},  & 0 \le x < 3\\
    \frac{5}{8},  & 3 \le x < 4\\
    1,            & x \ge 4
    \end{cases}
\]

\end{enumerate}

\end{document}
