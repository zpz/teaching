\documentclass[12pt]{article}
\usepackage{techart}
\begin{document}
\title{STAT 651 Midterm 1 Solution}
\maketitle

\section{}

\begin{enumerate}
\item[(a)]
\[
{13\choose 1}{4\choose 3}\cdot {12\choose 1}{4\choose 2}
= 13 \times 4 \times 12 \times 6
= 3744
\]

\item[(b)]
\[
{4\choose 1}{13\choose 5}
= 4\times \frac{13\cdot 12 \cdot 11 \cdot 10\cdot 9}
    {1\cdot 2 \cdot 3\cdot 4\cdot 5}
= 5148
\]

\item[(c)] Full house is less likely to occur than flush.
\end{enumerate}

\section{}

\begin{enumerate}
\item[(a)]
$S = \{1H, 1T, 2H, 2T, 3H, 3T, 4H, 4T\}$.
$S$ has 8 elements.

\item[(b)]
\begin{enumerate}
\item[i.]
Corresponding to the 8 elements of $S$ as listed above,
the values of $X$ are 1, 2, 2, 4, 3, 6, 4, 8.
Therefore,
\[
f_X(x) = \begin{cases}
    1/8, & x = 1\\
    1/4, & x = 2\\
    1/8, & x = 3\\
    1/4, & x = 4\\
    1/8, & x = 6\\
    1/8, & x = 8
\end{cases}
\]

\item[ii.]
\[
F_X(x) = \begin{cases}
    0,   & x < 1\\
    1/8, & 1 \le x < 2\\
    3/8, & 2 \le x < 3\\
    1/2, & 3 \le x < 4\\
    3/4, & 4 \le x < 6\\
    7/8, & 6 \le x < 8\\
    1,   & x \ge 8
\end{cases}
\]
Sketch is omitted.
\end{enumerate}
\end{enumerate}

\section{}
\begin{enumerate}
\item[(a)]
Independence suggests
$P(A\cap B) = P(A)P(B)$.

Mutual exclusiveness suggests
$P(A\cap B) = P(\emptyset) = 0$.

Hence $P(A)P(B) = 0$.
So, either $P(A) = 0$ or $P(B) = 0$.

\item[(b)]
No. It can be the case that both $A$ and $B$ are non-empty but
they are independent and mutually exclusive.

For example,
let $X$ be a number randomly drawn from the uniform distribution on
$[0,1]$.
Let $A = \{.1\}$ and $B = \{.2\}$.
Then $A$, $B$ are mutually exclusive and both have probability 0,
hence $P(A\cap B) = P(\emptyset) = 0 = P(A)P(B)$, that is,
they are independent.
But neither is empty.
\end{enumerate}

\section{}
Let $C$ be the event that the person is a child,
$A$ be the event that the person is an adult,
and $L$ be the event that the person likes the movie.
Then,
\[
P(C\given L)
= \frac{P(C \cap L)}{P(L)}
= \frac{P(C \cap L)}
    {P(L \cap C) + P(L\cap A)}
= \frac{1}{1 + \frac{P(L\given A)P(A)}{P(L\given C)P(C)}}
= \frac{1}{1 + \frac{(.90)(.60)}{(.95)(.40)}}
= .4130
\]

\section{}

\begin{enumerate}
\item[(a)]
\[
f_X(x) = \begin{cases}
    \frac{\diff F_X(x)}{\diff x} = 3e^{-3x}, & x \ge 0\\
    0, & x < 0
\end{cases}
\]
\item[(b)]
\begin{enumerate}
\item[i.]
Let $y = g(x) = e^x$, then $x = g^{-1}(y) = \log y$.
\[
f_Y(y)
= f_X(\log y) \Bigl|\frac{\diff \log y}{\diff y}\Bigr|
= 3e^{-3\log y} y^{-1}
= 3y^{-4},
\quad y \ge 1
\]
\item[ii.]
\[
\int_1^\infty 3y^{-4} \diff y
= -\int_1^\infty (-3)e^{-4} \diff y
= -\int_1^\infty \diff y^{-4}
= -y^{-4}\Bigm|_{y=1}^\infty
= 1
\]
\item[iii.]
\[
E Y
= \int_1^\infty y\cdot 3y^{-4}\diff y
= -\frac{3}{2}\int_1^\infty (-2)y^{-3} \diff y
= -\frac{3}{2} y^{-2}\Bigm|_{y=1}^\infty
= \frac{3}{2}
\]
\end{enumerate}
\end{enumerate}

\section{}
\begin{enumerate}
\item[(a)]
\[
E\{X^m\}
= p\cdot 1^m + (1-p)\cdot 0^m
= p
\]
\item[(b)]
\[
E\bigl\{e^{tX}\bigr\}
= p\cdot e^{t\cdot 1} + (1-p)\cdot e^{t\cdot 0}
= 1 - p + pe^t
\]
\item[(c)]
\[
E\bigl\{e^{-X}\bigr\}
= 1 - p + pe^{-1}
\]
\[
E\bigl\{e^{3X}\bigr\}
= 1 - p + pe^3
\]

\end{enumerate}

\end{document}
